\usepackage{cmap}
\usepackage[english, russian]{babel}
\usepackage{csquotes}
\usepackage{filecontents}
\selectlanguage{english}
\usepackage[
    backend=biber,
    style=gost-numeric,
    maxcitenames=3,
    maxbibnames=12,
    minnames=1,
    movenames=false,
    ibidtracker=false,
    autolang=other
    ]{biblatex}
\addbibresource{biblio/books.bib}
\usepackage{xunicode}
\usepackage{pdfpages}
\usepackage{xltxtra}


% Шрифты, xelatex
\defaultfontfeatures{Ligatures=TeX}
\setmainfont{Times New Roman} % Нормоконтроллеры хотят именно его
\newfontfamily\cyrillicfont{Times New Roman}
% \setsansfont{Liberation Sans} % Тут я его не использую, но если пригодится
\setmonofont{FreeMono} % Моноширинный шрифт для оформления кода

% % Формулы
\usepackage{mathtools,unicode-math} % Не совместим с amsmath
\setmathfont{XITS Math}             % Шрифт для формул: https://github.com/khaledhosny/xits-math
\numberwithin{equation}{section}    % Формула вида секция.номер

% Абзацы и списки
\usepackage{enumerate}   % Тонкая настройка списков
\usepackage{indentfirst} % Красная строка после заголовка
\usepackage{float}       % Расширенное управление плавающими объектами
\usepackage{multirow}    % Сложные таблицы

% Пути к каталогам с изображениями
\usepackage{graphicx} % Вставка картинок и дополнений
\graphicspath{{images/}}

% % Формат подрисуночных записей
\usepackage{chngcntr}

% % Сбрасываем счетчик таблиц и рисунков в каждой новой главе
% \counterwithin{figure}{section}
\counterwithin{table}{section}

% % Гиперссылки
\usepackage{hyperref}
\hypersetup{
    colorlinks, urlcolor={black}, % Все ссылки черного цвета, кликабельные
    linkcolor={black}, citecolor={black}, filecolor={black},
    pdfauthor={Александр Коротков},
    pdftitle={Анализ алгоритмов глубокого машинного обучения в задачах распознавания изображений}
}

% Оформление библиографии и подрисуночных записей через точку
\makeatletter
% \renewcommand*{\@biblabel}[1]{\hfill#1.}
\renewcommand*\l@section{\@dottedtocline{1}{1em}{1em}}
\def\redeflsection{\def\l@section{\@dottedtocline{1}{0em}{10em}}}
\makeatother

\renewcommand{\baselinestretch}{1.5} % Полуторный межстрочный интервал
\parindent 1.27cm % Абзацный отступ

\sloppy             % Избавляемся от переполнений
\hyphenpenalty=1000 % Частота переносов
\clubpenalty=10000  % Запрещаем разрыв страницы после первой строки абзаца
\widowpenalty=10000 % Запрещаем разрыв страницы после последней строки абзаца

% Отступы у страниц
\usepackage{geometry}
\geometry{left=3cm}
\geometry{right=2cm}
\geometry{top=2cm}
\geometry{bottom=2cm}

% Списки
\usepackage{enumitem}
\setlist[enumerate,itemize]{leftmargin=12.7mm} % Отступы в списках

\makeatletter
    \AddEnumerateCounter{\asbuk}{\@asbuk}{м)}
\makeatother
\setlist{nolistsep}                           % Нет отступов между пунктами списка
% \renewcommand{\labelitemi}{--}                % Маркер списка --
\renewcommand{\labelenumi}{\asbuk{enumi})}    % Список второго уровня
\renewcommand{\labelenumii}{\arabic{enumii})} % Список третьего уровня

% Содержание
\usepackage{tocloft}
\renewcommand{\cfttoctitlefont}{\hspace*{\fill}\bfseries}
\renewcommand{\cftaftertoctitle}{\hspace*{\fill}}
\setcounter{tocdepth}{3}                            % Глубина оглавления, до subsubsection


% Формат подрисуночных надписей
\RequirePackage{caption}
\DeclareCaptionLabelSeparator{deffis}{ -- } % Разделитель
\DeclareCaptionLabelFormat{rtable}{\hfill #1~#2}
\captionsetup[figure]{justification=centering, labelsep=deffis, labelfont=bf, format=plain} % Подпись рисунка по центру
\captionsetup[table]{justification=centerlast, labelsep=newline, textfont=bf, labelformat=rtable, format=hang, position=above, margin=0em} 
\addto\captionsrussian{\renewcommand{\figurename}{Рис.}} % Имя фигуры

% Пользовательские функции
\newcommand{\addimg}[4]{ % Добавление одного рисунка
    \begin{figure}
        \centering
        \includegraphics[width=#2\linewidth]{#1}
        \caption{#3} \label{#4}
    \end{figure}
}
\newcommand{\addimghere}[4]{ % Добавить рисунок непосредственно в это место
    \begin{figure}[H]
        \centering
        \includegraphics[width=#2\linewidth]{#1}
        \caption{#3} \label{#4}
    \end{figure}
}
\newcommand{\addtwoimghere}[5]{ % Вставка двух рисунков
    \begin{figure}[H]
        \centering
        \includegraphics[width=#2\linewidth]{#1}
        \hfill
        \includegraphics[width=#3\linewidth]{#2}
        \caption{#4} \label{#5}
    \end{figure}
}


% % Заголовки секций в оглавлении в верхнем регистре
% \usepackage{textcase}
% \makeatletter
% \let\oldcontentsline\contentsline
% \def\contentsline#1#2{
%     \expandafter\ifx\csname l@#1\endcsname\l@section
%         \expandafter\@firstoftwo
%     \else
%         \expandafter\@secondoftwo
%     \fi
%     {\oldcontentsline{#1}{#2}}
%     {\oldcontentsline{#1}{#2}}
% }
% \makeatother

% Оформление заголовков
\usepackage[compact, explicit]{titlesec}
\titleformat{\section}{}{}{12.5mm}{\bfseries\centering{\thesection\quad{#1}}\vspace{1.5em}}
\titleformat{\subsection}[block]{\bfseries\centering\vspace{1em}}{}{12.5mm}{\thesubsection\quad#1\vspace{1em}}
\titleformat{\subsubsection}[block]{\bfseries\centering\vspace{1em}\normalsize}{}{12.5mm}{\thesubsubsection\quad#1\vspace{1em}}
\titleformat{\paragraph}[block]{\normalsize\centering}{}{12.5mm}{#1}
% Секции без номеров (введение, заключение...), вместо section*{}
\newcommand{\anonsection}[1]{
    \phantomsection % Корректный переход по ссылкам в содержании
    \paragraph{\bfseries{{#1}}\vspace{1em}}
    \addcontentsline{toc}{section}{#1}
}

% Секция для аннотации (она не включается в содержание)
\newcommand{\annotation}[1]{
    \paragraph{\bfseries{{#1}}\vspace{1em}}
}

% Секции для приложений
\newcommand{\appsection}[1]{
    \phantomsection    
    \paragraph{\hfill\bfseries{{#1}}}
    \addcontentsline{toc}{section}{{#1}}
}

\usepackage{lastpage} % Подсчет количества страниц

% \setcounter{page}{2}  % Начало нумерации страниц

% \usepackage{lastbib} % Кол-во источников

\usepackage{totcount}
\usepackage[figure, table]{totalcount}

\newtotcounter{citnum} %% Счётчик библиографии
\AtEveryBibitem{\stepcounter{citnum}}

\regtotcounter{section}

\usepackage{listings} % Оформление исходного кода
\lstset{
    basicstyle=\small\ttfamily, % Размер и тип шрифта
    breaklines=true,            % Перенос строк
    tabsize=2,                  % Размер табуляции
    frame=single,               % Рамка
    literate={--}{{-{}-}}2,     % Корректно отображать двойной дефис
    literate={---}{{-{}-{}-}}3, % Корректно отображать тройной дефис
    language=Python,            % Язык программирования
    numbers=left,               % Нумерация строк
}

% Графики и таблицы
\usepackage{tikz}
\usepackage{pgfplots}
\usetikzlibrary{matrix, positioning}


\usepackage{array, makecell}
\usepackage{xfrac}

% Нейросети
\usepackage{neuralnetwork}

\newcommand{\x}[2]{$x_#2$}
\newcommand{\y}[2]{$\hat{y}_#2$}
\newcommand{\hfirst}[2]{\small $h_#2$}