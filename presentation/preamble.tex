% \usepackage[utf8]{inputenc}
\usepackage[russian]{babel}
% \usepackage{csquotes}

\usepackage{fontspec}
% \usepackage{xunicode}
% \usepackage{pdfpages}
% \usepackage{xltxtra}

% \defaultfontfeatures{Ligatures=TeX}
\setmainfont{Times New Roman} % Нормоконтроллеры хотят именно его
\usefonttheme{serif}

\usepackage{tikz} % Для создания рисунков с помощью tikz
\usepackage{listings} % Для листингов программ

\usecolortheme{whale}
% \usecolortheme{dolphin}
% \usecolortheme{warsaw}
%\usetheme{Madrid}

\usepackage{graphicx} % Вставка картинок и дополнений
\graphicspath{{extra/}}
\usepackage{caption}
\captionsetup[figure]{labelformat=empty, labelsep=none}

\useinnertheme{rounded} % Скругляем углы блоков
\useoutertheme{infolines} % Добавляем колонтитулы
\setbeamertemplate{headline}{} % Убираем верхний колонтитул
\setbeamertemplate{navigation symbols}{} % Убираем навигационные символы
\usefonttheme[onlysmall]{structurebold} % жирный шрифт в колонтитулах
%\setbeamerfont*{frametitle}{size=\normalsize,series=\bfseries}
\setbeamercovered{transparent} % uncover будет рисовать закрытые части текста серым
%\setbeamercovered{invisible} % а так -- невидимым
% Определяем цвет структурных элементов
\definecolor{unicover}{rgb}{0.10,0.35,0.62} 
\setbeamercolor{structure}{fg=unicover}
% Добавляем логотип университета
\pgfdeclareimage[height=1.5cm]{university-logo}{LogoYarSU.png}
\logo{\pgfuseimage{university-logo}}

% Делим нижний колонтитул на три части: автор, название, число слайдов
\newcommand{\makefootline}[3]{
	\setbeamertemplate{footline}{
		\leavevmode%
		\hbox{%
			\begin{beamercolorbox}[wd=#1\paperwidth,ht=2.25ex,dp=1ex,center]{author in head/foot}
				\usebeamerfont{author in head/foot}%
				\insertshortauthor
			\end{beamercolorbox}
			\begin{beamercolorbox}[wd=#2\paperwidth,ht=2.25ex,dp=1ex,center]{title in head/foot}
				\usebeamerfont{title in head/foot}\insertshorttitle
			\end{beamercolorbox}
			\begin{beamercolorbox}[wd=#3\paperwidth,ht=2.25ex,dp=1ex,right]{date in head/foot}
				\insertframenumber{} / \inserttotalframenumber\hspace*{2ex}
			\end{beamercolorbox}
		}%
	}
}
\makefootline{.2}{.7}{.1} % Сумма длин = 1

% Делаем блоки прозрачными, чтобы они не закрывали логотип
\setbeamertemplate{blocks}[default]
\setbeamercolor{block title}{bg=}
\setbeamercolor{block body}{bg=}

% Графики и таблицы
\usepackage{tikz}
\usepackage{pgfplots}
\usetikzlibrary{matrix, positioning}
\usepackage{bm}
\usepackage{relsize}

\usepackage{array, makecell}
\usepackage{xfrac}
\usepackage{multirow}

\usepackage[labelformat=empty, labelsep=none]{subcaption}
\newcommand{\addthreeimghere}[8]{ % Вставка трех рисунков
    \begin{figure}[H]   
        \centering    
        \subcaptionbox{#2}{\includegraphics[width=.3\textwidth]{#1}}   
        % \hfill
        \subcaptionbox{#4}{\includegraphics[width=.3\textwidth]{#3}}
        % \hfill
        \subcaptionbox{#6}{\includegraphics[width=.3\textwidth]{#5}}
        \caption{#7}
        \label{#8}
    \end{figure}
}
