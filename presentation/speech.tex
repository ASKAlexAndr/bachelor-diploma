[1]
Здравствуйте, уважаемые преподаватели!
Вашему вниманию предлагается доклад Короткова Александра Сергеевича на тему выпускной квалификационной работы «Анализ алгоритмов глубокого машинного обучения  в задачах распознавания изображений».
[2]
В ходе работы были поставлены цели и задачи, которые вы можете видеть на слайде.
В процессе решения данных задач были изучены вопросы классификации изображений при помощи глубоких нейронных сетей, в частности сверточных нейронных сетей. 
[3]
Под СНС понимают такие сети, которые используют в своих основных слоях операцию свертки, суть которой схематично изображена на слайде. Было проведено изучение и сравнение различных архитектур сверточных нейронных сетей, таких как VGG, Inception, ResNet и DenseNet и проведен их анализ.   
[4]
Исследования показывают, что большинство рентгеновских снимков пациентов с COVID-19 содержат спецефические аномалии, такие как двустороннее ретикулярное узловое затемнение или синдром матового стекла, что позволяет визуально отличить заболевание от других видов пневмоний. Несмотря на то, что рентген обладает меньшей чуствительностью чем КТ или ПЦР, это гораздо более доступный и быстрый метод диагностики, что является существенным критерием в период пандемии.
[5]
В ходе работы были разработаны, обучены и протестированы модели Inception-V3, ResNet-152 V2 и DenseNet-201, для решения задачи диагностики пневмонии, и в частности COVID-19, по рентгеновским снимкам грудной клетки. На графиках вы можете видеть точность, которую показывали сети в процессе обучения и тестрования. Как вы можете видеть, лучшие результаты показала сеть Inception третьей версии.


