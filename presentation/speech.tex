[1]
Здравствуйте, уважаемые члены комиссии!
Вашему вниманию предлагается доклад на тему выпускной квалификационной работы «Анализ алгоритмов глубокого машинного обучения  в задачах распознавания изображений».

[2]
В ходе данной работы были поставлены цели и задачи, которые вы можете видеть на слайде. В процессе решения данных задач были изучены вопросы классификации изображений при помощи глубоких сверточных нейронных сетей. Также была сформирована база рентгеновских снимков пациентов и обучены различные модели сверточных нейронных сетей для решения задачи классификации собранных изображений.

[3]
Под СНС понимают такие сети, которые используют в своих основных слоях операцию свертки, суть которой схематично изображена на слайде. Было проведено изучение и сравнение различных архитектур сверточных нейронных сетей, таких как VGG, Inception, ResNet и DenseNet. В таблице указаны результаты тестирования на датасете ImageNet, которое включает в себя 1000 классов, свыше 1.3 млн. тренировочных и 50 тыс. проверочных изображений. Toчность сетей измерялась в двух вариантах Top-1 и Top-5. Top-1 означает, что сеть правильно классифицировала изображение, а Top-5 - что правильный класс принадлежит к пяти самых высоким прогнозам сети. Дальнейшие исследования проводились на выделенных в таблице сетях.

[4]
Анализ качества сетей проводился по следующим метрикам.
Точность (Precision) обозначает долю правильно идентифицированных объектов класса относительно всех объектов причисленных к этому классу. Полнота (Recall) показывет долю найденных сетью элементов класса относительно всех элементов этого класса.
F1 объединяет точность и полноту вычисляя их гармоническое среднее.

[5]
Контроль обучения нейронной сети требует введения некоторой численной оценки качества её работы. Для этого вводится функция потерь, которая вычисляет разницу между правильными и полученными результатами. В данной работе при обучении моделей использовалась категориальная перекрестная энтропия.

[6]
Для практической части работы была выбрана актуальная тема диагностики COVID-19.
Исследования показывают, что большинство рентгеновских снимков пациентов с COVID-19 содержат специфические аномалии, такие как двустороннее ретикулярное узловое затемнение или синдром матового стекла, что позволяет визуально отличить заболевание от других видов пневмоний. Несмотря на то, что рентген обладает меньшей чувствительностью чем, к примеру Компьютерная томография, это гораздо более доступный и быстрый метод диагностики, что является существенным критерием в период пандемии.
A) неоднородные уплотнения, В) плевральный экссудат, С) перихилярная локализация и D) периферическая локализация.
[7]
Использование нейронных сетей зачастую требует проведения дополнительных исследований, с целью выбора оптимальных параметров различных частей алгоритмов. Предварительная обработка входных данных может существенно повлиять на ход обучения моделей. В рамках исследования были рассмотрены два варианта предварительной обработки изображений: масштабирование значений и нормализацией.
В работе были протестированы указанные модели нейронных сетей со стандартными параметрами. 

[8]
В таблицах показаны результаты проведенных тестов: значения функции ошибки, метрик точности (precision) и полноты (recall)  по итогам обучения и валидации (val_). Как вы можете видеть предварительное масштабирование значений демонстрирует более высокие результаты по всем параметрам. При этом сеть ResNet-50 дала минимальное значение функции ошибки, а сеть inception самую высокую точность и полноту при валидации моделей.

[9]
Далее было проведено исследование с целью выбора алгоритма оптимизации функции потерь, цель которого минимизация данной функции. Для объективности тестирования обучение всех моделей проходило одинакого: по 10 эпох и с размерами входных изображений 500x500 пикселей. Анализ проходил по метрикам точности, полноты и F1-мере. Для выбора оптимизатора рассматривались различные алгоритмы: стохастического градиентного спуска, RMSprop и Adam. 

[10]
В таблицах указаны результаты которые показали сети при валидации по итогу проведенных экспериментов. Как вы можете видеть сеть ResNet с алгоритмом RMSprop по F1-мере лучше других классифицирует изображения с COVID-19, однако хуже справляется с другими классами. При этом сеть Inception вместе с SGD в среднем по всем классам показывает самые высокие значения метрик.    

[11]
В результате данной работы было проведено исследование алгоритмов глубокого машинного обучения. Было структурировано свыше 14 тыс. изображений рентгеновских снимков пациентов. Проведено сравнение двух вариантов предварительной обработки изображений: масштабирование значений (на 1/255) и их нормализация. В ходе сравнения масштабирование показало меньшие значения ошибок на всех сетях.
Также проведено тестирование моделей Inception V3, ResNet­-50 V2 и DenseNet-201 и оптимизаторов SGD, RMSprop и Adam, с целью нахождения наиболее эффективной связки. В результате экспериментов можно сделать вывод, что сеть Inception V3 с алгоритмом SGD наиболее предпочтительна для решения задачи классификации пневмонии и COVID-19 по рентгеновским снимкам грудной клетки.

[12]
Спасибо за внимание!

