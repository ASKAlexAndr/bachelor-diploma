% % !TEX program = xelatex

% Author: Alexandr Korotkov
% https://github.com/ASKAlexAndr/bachelor-thesis

\documentclass[aspectratio=169]{beamer}
\hypersetup{pdfpagemode=FullScreen}
\title{Анализ алгоритмов глубокого машинного обучения  в задачах распознавания изображений}

\author[Коротков А.С.]{Александр Сергеевич Коротков}

\institute[]{Научный руководитель: Д.В. Матвеев}

\date{16.06.2020}

% \usepackage[utf8]{inputenc}
\usepackage[russian]{babel}
% \usepackage{csquotes}

\usepackage{fontspec}
% \usepackage{xunicode}
% \usepackage{pdfpages}
% \usepackage{xltxtra}

% \defaultfontfeatures{Ligatures=TeX}
\setmainfont{Times New Roman} % Нормоконтроллеры хотят именно его
\usefonttheme{serif}

\usepackage{tikz} % Для создания рисунков с помощью tikz
\usepackage{listings} % Для листингов программ

% \usecolortheme{whale}
\usecolortheme{dolphin} 
% \usetheme{warsaw}
%\usetheme{Madrid}

\useinnertheme{rounded} % Скругляем углы блоков
\useoutertheme{infolines} % Добавляем колонтитулы
\setbeamertemplate{headline}{} % Убираем верхний колонтитул
\setbeamertemplate{navigation symbols}{} % Убираем навигационные символы
\usefonttheme[onlysmall]{structurebold} % жирный шрифт в колонтитулах
%\setbeamerfont*{frametitle}{size=\normalsize,series=\bfseries}
\setbeamercovered{transparent} % uncover будет рисовать закрытые части текста серым
%\setbeamercovered{invisible} % а так -- невидимым
% Определяем цвет структурных элементов
\definecolor{unicover}{rgb}{0.10,0.35,0.62} 
\setbeamercolor{structure}{fg=unicover}
% Добавляем логотип университета
\pgfdeclareimage[height=1.5cm]{university-logo}{LogoYarSU.png}
\logo{\pgfuseimage{university-logo}}

% Делим нижний колонтитул на три части: автор, название, число слайдов
\newcommand{\makefootline}[3]{
	\setbeamertemplate{footline}{
		\leavevmode%
		\hbox{%
			\begin{beamercolorbox}[wd=#1\paperwidth,ht=2.25ex,dp=1ex,center]{author in head/foot}
				\usebeamerfont{author in head/foot}%
				\insertshortauthor
			\end{beamercolorbox}
			\begin{beamercolorbox}[wd=#2\paperwidth,ht=2.25ex,dp=1ex,center]{title in head/foot}
				\usebeamerfont{title in head/foot}\insertshorttitle
			\end{beamercolorbox}
			\begin{beamercolorbox}[wd=#3\paperwidth,ht=2.25ex,dp=1ex,right]{date in head/foot}
				\insertframenumber{} / \inserttotalframenumber\hspace*{2ex}
			\end{beamercolorbox}
		}%
	}
}
\makefootline{.2}{.7}{.1} % Сумма длин = 1

% Делаем блоки прозрачными, чтобы они не закрывали логотип
\setbeamertemplate{blocks}[default]
\setbeamercolor{block title}{bg=}
\setbeamercolor{block body}{bg=}

 % Подключаем преамбулу

\begin{document}
\maketitle
\begin{frame}{Цели и задачи работы}
    \textbf{Цель:} Изучить и проанализировать применение алгоритмов глубокого машинного обучения в задачах распознавания изображений \\
    \textbf{Задачи: }
    \begin{itemize}
        \item Изучить теоретический материал про обучение глубоких нейронных сетей и их применение в классификации изображений
        \item Изучить документацию библиотеки Tensorflow.
        \item Изучить вопрос диагностики COVID-19 по рентгеновским снимкам грудной клетки.
        \item Разработать и обучить различные модели сверточных нейронных сетей на наборе рентгеновских снимков.
        \item Сравнить точности работы реализованных нейронных сетей.
    \end{itemize}
\end{frame}

\begin{frame}{Сверточные нейронные сети}
    \begin{columns}[T]
        \begin{column}{.4\paperwidth}
            \centering
            \begin{figure}[H]
\centering
\begin{tikzpicture}[scale=0.8,  every node/.style={scale=0.8}]

	\matrix (mtr) [matrix of nodes,row sep=-\pgflinewidth, nodes={draw}]
	{
		0 & 1 & 1 & |[fill=red!30]| 1 & |[fill=red!30]| 0 & |[fill=red!30]| 0 & 0\\
		0 & 0 & 1 & |[fill=red!30]| 1 & |[fill=red!30]| 1 & |[fill=red!30]| 0 & 0\\
		0 & 0 & 0 & |[fill=red!30]| 1 & |[fill=red!30]| 1 & |[fill=red!30]| 1 & 0\\
		0 & 0 & 0 & 1 & 1 & 0 & 0\\
		0 & 0 & 1 & 1 & 0 & 0 & 0\\
		0 & 1 & 1 & 0 & 0 & 0 & 0\\
		1 & 1 & 0 & 0 & 0 & 0 & 0\\
	};

	\draw[very thick, red] (mtr-1-4.north west) rectangle (mtr-3-6.south east);

	\matrix (K) [right=0.2em of mtr,matrix of nodes,row sep=-\pgflinewidth, nodes={draw, fill=blue!30}]
	{
		1 & 0 & 1 \\
		0 & 1 & 0 \\
		1 & 0 & 1 \\
	};

	\matrix (ret) [right=0.2em of K,matrix of nodes,row sep=-\pgflinewidth, nodes={draw}]
	{
		1 & 4 & 3 & |[fill=green!30]| 4 & 1\\
		1 & 2 & 4 & 3 & 3\\
		1 & 2 & 3 & 4 & 1\\
		1 & 3 & 3 & 1 & 1\\
		3 & 3 & 1 & 1 & 0\\
	};

	\draw[very thick, green] (ret-1-4.north west) rectangle (ret-1-4.south east);

	\draw[densely dotted, blue, thick] (mtr-1-4.north west) -- (K-1-1.north west);
	\draw[densely dotted, blue, thick] (mtr-3-4.south west) -- (K-3-1.south west);
	\draw[densely dotted, blue, thick] (mtr-1-6.north east) -- (K-1-3.north east);
	\draw[densely dotted, blue, thick] (mtr-3-6.south east) -- (K-3-3.south east);

	\draw[densely dotted, green, thick] (ret-1-4.north west) -- (K-1-1.north west);
	\draw[densely dotted, green, thick] (ret-1-4.south west) -- (K-3-1.south west);
	\draw[densely dotted, green, thick] (ret-1-4.north east) -- (K-1-3.north east);
	\draw[densely dotted, green, thick] (ret-1-4.south east) -- (K-3-3.south east);

	\matrix (K) [right=0.2em of mtr,matrix of nodes,row sep=-\pgflinewidth, nodes={draw, fill=blue!10}]
	{
		1 & 0 & 1 \\
		0 & 1 & 0 \\
		1 & 0 & 1 \\
	};

	\draw[very thick, blue] (K-1-1.north west) rectangle (K-3-3.south east);
\end{tikzpicture}
\caption{Операция свертки} \label{convolution}
\end{figure}    
        \end{column}
        \begin{column}{.5\paperwidth}
            \begin{table}[H]
    \centering
    \caption{Результаты тестирования моделей} \label{conv-test}
    \begin{tabular}{|c|c|c|c|}
      \hline    
      Сеть         & Кол-во параметров   & Top-1     & Top-5     \\
      \hline
      VGG-16       & 138 357 544         & 71.3\%	& 90.1\%    \\
      \hline
      VGG-19       & 143 667 240         & 71.3\%	& 90.0\%    \\
      \hline
      Inception V3 & 23 851 784          & 77.9\%	& 93.7\%    \\
      \hline
      % ResNet-50    & 25 636 712	         & 74.9\%	& 92.1\%    \\
      % \hline
      % ResNet-101   & 44 707 176	         & 76.4\%	& 92.8\%    \\
      % \hline
      % ResNet-152   & 60 419 944		       & 76.6\%	& 93.1\%    \\
      % \hline
      ResNet-50 V2 & 25 613 800          & 76.0\%	& 93.0\%    \\
      \hline
      ResNet-101 V2 & 44 675 560	       & 77.2\%	& 93.8\%    \\
      \hline
      ResNet-152 V2 & 60 380 648	       & 78.0\%	& 94.2\%    \\
      \hline
      DenseNet-121 & 8 062 504           & 75.0\%	& 92.3\%    \\
      \hline
      DenseNet-169 & 14 307 880          & 76.2\%	& 93.2\%    \\
      \hline
      DenseNet-201 & 20 242 984          & 77.3\%	& 93.6\%    \\
      \hline
    \end{tabular}
  \end{table}
        \end{column}
    \end{columns}   
\end{frame}

\begin{frame}{Задача диагностики COVID-19}
    \addthreeimghere{xray-normal}{Норма}{xray-pneumonia}{Пневмония}{xray-covid}{COVID-19}{Рентгеновские снимки грудных клеток}{xrays}
\end{frame}

\begin{frame}{Обучение сетей}
    \begin{columns}[T]
        \begin{column}{.4\paperwidth}
            \begin{figure}
                \centering
                \includegraphics[width=\textwidth]{train_accuracy.pdf} 
                \caption{Обучение}
            \end{figure}            
        \end{column}
        \begin{column}{.4\paperwidth}
            \begin{figure}
                \centering                
                \includegraphics[width=\textwidth]{val_accuracy.pdf}
                \caption{Тестирование}
            \end{figure} 
        \end{column}
    \end{columns}   
\end{frame}
\begin{frame}{Заключение}
    Итоги:\\
    \begin{itemize}
        \item Глубокое обучение эффективно справляется с задачей классификации изображений. 
        \item Современные модели нейронных сетей обладают большим потенциалом для .
        \item Разработаны и обучены модели для диагностики COVID-19.
    \end{itemize}

    \vspace{2em}    
    Что дальше?
    \begin{itemize}
        \item Адаптировать модели для задачи.
        \item Увеличить базу данных и повысить длительность обучения.
        \item Проанализировать работу сетей по другим метрикам.
    \end{itemize}
\end{frame}
\begin{frame}
    \centering\Huge
    Спасибо за внимание!
\end{frame}
\end{document}