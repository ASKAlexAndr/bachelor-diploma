% % !TEX program = xelatex

% Author: Alexandr Korotkov
% https://github.com/ASKAlexAndr/bachelor-thesis

\documentclass{beamer}

\title{Анализ алгоритмов глубокого машинного обучения  в задачах распознавания изображений}

\author[Коротков А.С.]{Александр Сергеевич Коротков}

\institute[]{Научный руководитель: Д.\,В.~Матвеев}

\date{16.06.2020}

% \usepackage[utf8]{inputenc}
\usepackage[russian]{babel}
% \usepackage{csquotes}

\usepackage{fontspec}
% \usepackage{xunicode}
% \usepackage{pdfpages}
% \usepackage{xltxtra}

% \defaultfontfeatures{Ligatures=TeX}
\setmainfont{Times New Roman} % Нормоконтроллеры хотят именно его
\usefonttheme{serif}

\usepackage{tikz} % Для создания рисунков с помощью tikz
\usepackage{listings} % Для листингов программ

% \usecolortheme{whale}
\usecolortheme{dolphin} 
% \usetheme{warsaw}
%\usetheme{Madrid}

\useinnertheme{rounded} % Скругляем углы блоков
\useoutertheme{infolines} % Добавляем колонтитулы
\setbeamertemplate{headline}{} % Убираем верхний колонтитул
\setbeamertemplate{navigation symbols}{} % Убираем навигационные символы
\usefonttheme[onlysmall]{structurebold} % жирный шрифт в колонтитулах
%\setbeamerfont*{frametitle}{size=\normalsize,series=\bfseries}
\setbeamercovered{transparent} % uncover будет рисовать закрытые части текста серым
%\setbeamercovered{invisible} % а так -- невидимым
% Определяем цвет структурных элементов
\definecolor{unicover}{rgb}{0.10,0.35,0.62} 
\setbeamercolor{structure}{fg=unicover}
% Добавляем логотип университета
\pgfdeclareimage[height=1.5cm]{university-logo}{LogoYarSU.png}
\logo{\pgfuseimage{university-logo}}

% Делим нижний колонтитул на три части: автор, название, число слайдов
\newcommand{\makefootline}[3]{
	\setbeamertemplate{footline}{
		\leavevmode%
		\hbox{%
			\begin{beamercolorbox}[wd=#1\paperwidth,ht=2.25ex,dp=1ex,center]{author in head/foot}
				\usebeamerfont{author in head/foot}%
				\insertshortauthor
			\end{beamercolorbox}
			\begin{beamercolorbox}[wd=#2\paperwidth,ht=2.25ex,dp=1ex,center]{title in head/foot}
				\usebeamerfont{title in head/foot}\insertshorttitle
			\end{beamercolorbox}
			\begin{beamercolorbox}[wd=#3\paperwidth,ht=2.25ex,dp=1ex,right]{date in head/foot}
				\insertframenumber{} / \inserttotalframenumber\hspace*{2ex}
			\end{beamercolorbox}
		}%
	}
}
\makefootline{.2}{.7}{.1} % Сумма длин = 1

% Делаем блоки прозрачными, чтобы они не закрывали логотип
\setbeamertemplate{blocks}[default]
\setbeamercolor{block title}{bg=}
\setbeamercolor{block body}{bg=}

 % Подключаем преамбулу

\begin{document}
\maketitle
\begin{frame}{Цели и задачи работы}
    \textbf{Цель:} Изучить и проанализировать применение алгоритмов глубокого машинного обучения в задачах распознавания изображений \\
    \textbf{Задачи: }
    \begin{itemize}
        \item Изучить теоретический материал про обучение глубоких нейронных сетей и их применение в классификации изображений
        \item Изучить документацию библиотеки Tensorflow.
        \item Изучить вопрос диагностирования COVID-19 по рентгеновским снимкам грудной клетки.
        \item Разработать и обучить различные модели сверточных нейронных сетей на наборе рентгеновских снимков.
        \item Сравнить точности работы реализованных нейронных сетей.
    \end{itemize}
\end{frame}

\begin{frame}{Изученные модели}

    
\end{frame}

\begin{frame}{Обучение сетей}
    \begin{columns}[T]
        \begin{column}{.5\paperwidth}
            \begin{figure}
                \centering
                \includegraphics[width=\textwidth]{train_accuracy.pdf} 
                \caption{Обучение}
            \end{figure}            
        \end{column}
        \begin{column}{.5\paperwidth}
            \begin{figure}
                \centering                
                \includegraphics[width=\textwidth]{val_accuracy.pdf}
                \caption{Тестирование}
            \end{figure} 
        \end{column}
    \end{columns}   
\end{frame}
\note{
    Уважаемый председатель и члены комиссии!
Вашему вниманию предлагается доклад Короткова Александра Сергеевича на тему выпускной квалификационной работы «Анализ алгоритмов глубокого машинного обучения  в задачах распознавания изображений».
В ходе работы передо мной были поставлены цели и задачи, которые вы можете видеть на слайде. 
В процессе решения данных задач были изучены вопросы классификации изображений при помощи глубоких нейронных сетей. Были рассмотрены современные архитектуры сверточных нейронных сетей, такие как VGG, Inception, ResNet и DenseNet и проведен их анализ. 
Также были разработаны, обучены и протестированы указанные модели для решения задачи диагностики пневмонии, и в частности COVID-19, по рентгеновским снимкам грудной клетки.

По итогам обучения сети показали следующие результаты []

}
\begin{frame}
    
\end{frame}
\end{document}