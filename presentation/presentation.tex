% % !TEX program = xelatex

% Author: Alexandr Korotkov
% https://github.com/ASKAlexAndr/bachelor-thesis

\documentclass[aspectratio=169]{beamer}
\hypersetup{pdfpagemode=FullScreen}
\title{Анализ алгоритмов глубокого машинного обучения  в задачах распознавания изображений}

\author[Коротков А.С.]{Александр Сергеевич Коротков}

\institute[]{Научный руководитель: Д.В. Матвеев}

\date{29.06.2020}

% \usepackage[utf8]{inputenc}
\usepackage[russian]{babel}
% \usepackage{csquotes}

\usepackage{fontspec}
% \usepackage{xunicode}
% \usepackage{pdfpages}
% \usepackage{xltxtra}

% \defaultfontfeatures{Ligatures=TeX}
\setmainfont{Times New Roman} % Нормоконтроллеры хотят именно его
\usefonttheme{serif}

\usepackage{tikz} % Для создания рисунков с помощью tikz
\usepackage{listings} % Для листингов программ

% \usecolortheme{whale}
\usecolortheme{dolphin} 
% \usetheme{warsaw}
%\usetheme{Madrid}

\useinnertheme{rounded} % Скругляем углы блоков
\useoutertheme{infolines} % Добавляем колонтитулы
\setbeamertemplate{headline}{} % Убираем верхний колонтитул
\setbeamertemplate{navigation symbols}{} % Убираем навигационные символы
\usefonttheme[onlysmall]{structurebold} % жирный шрифт в колонтитулах
%\setbeamerfont*{frametitle}{size=\normalsize,series=\bfseries}
\setbeamercovered{transparent} % uncover будет рисовать закрытые части текста серым
%\setbeamercovered{invisible} % а так -- невидимым
% Определяем цвет структурных элементов
\definecolor{unicover}{rgb}{0.10,0.35,0.62} 
\setbeamercolor{structure}{fg=unicover}
% Добавляем логотип университета
\pgfdeclareimage[height=1.5cm]{university-logo}{LogoYarSU.png}
\logo{\pgfuseimage{university-logo}}

% Делим нижний колонтитул на три части: автор, название, число слайдов
\newcommand{\makefootline}[3]{
	\setbeamertemplate{footline}{
		\leavevmode%
		\hbox{%
			\begin{beamercolorbox}[wd=#1\paperwidth,ht=2.25ex,dp=1ex,center]{author in head/foot}
				\usebeamerfont{author in head/foot}%
				\insertshortauthor
			\end{beamercolorbox}
			\begin{beamercolorbox}[wd=#2\paperwidth,ht=2.25ex,dp=1ex,center]{title in head/foot}
				\usebeamerfont{title in head/foot}\insertshorttitle
			\end{beamercolorbox}
			\begin{beamercolorbox}[wd=#3\paperwidth,ht=2.25ex,dp=1ex,right]{date in head/foot}
				\insertframenumber{} / \inserttotalframenumber\hspace*{2ex}
			\end{beamercolorbox}
		}%
	}
}
\makefootline{.2}{.7}{.1} % Сумма длин = 1

% Делаем блоки прозрачными, чтобы они не закрывали логотип
\setbeamertemplate{blocks}[default]
\setbeamercolor{block title}{bg=}
\setbeamercolor{block body}{bg=}

 

\begin{document}
\maketitle
\begin{frame}{Цели и задачи работы}
    \textbf{Цель:} Изучить и проанализировать применение алгоритмов глубокого машинного обучения  в задачах обработки рентгеновских снимков у больных с подозрением на COVID-19. \\
    \textbf{Задачи: }
    \begin{itemize}
        \item Изучить теоретический материал про обучение глубоких нейронных сетей и их применение в классификации изображений.
        \item Изучить документацию библиотеки Tensorflow.
        \item Изучить вопрос диагностики COVID-19 по рентгеновским снимкам грудной клетки.
        \item Разработать и обучить различные модели сверточных нейронных сетей на наборе рентгеновских снимков.
        \item Сравнить применение различных подходов предварительной обрабоки изображений.
        \item Сравнить результаты работы реализованных нейронных сетей.
    \end{itemize}
\end{frame}

\begin{frame}{Сверточные нейронные сети}
    \begin{columns}[T]
        \begin{column}{.4\paperwidth}
            \centering
            \begin{figure}[H]
\centering
\begin{tikzpicture}[scale=0.8,  every node/.style={scale=0.8}]

	\matrix (mtr) [matrix of nodes,row sep=-\pgflinewidth, nodes={draw}]
	{
		0 & 1 & 1 & |[fill=red!30]| 1 & |[fill=red!30]| 0 & |[fill=red!30]| 0 & 0\\
		0 & 0 & 1 & |[fill=red!30]| 1 & |[fill=red!30]| 1 & |[fill=red!30]| 0 & 0\\
		0 & 0 & 0 & |[fill=red!30]| 1 & |[fill=red!30]| 1 & |[fill=red!30]| 1 & 0\\
		0 & 0 & 0 & 1 & 1 & 0 & 0\\
		0 & 0 & 1 & 1 & 0 & 0 & 0\\
		0 & 1 & 1 & 0 & 0 & 0 & 0\\
		1 & 1 & 0 & 0 & 0 & 0 & 0\\
	};

	\draw[very thick, red] (mtr-1-4.north west) rectangle (mtr-3-6.south east);

	\matrix (K) [right=0.2em of mtr,matrix of nodes,row sep=-\pgflinewidth, nodes={draw, fill=blue!30}]
	{
		1 & 0 & 1 \\
		0 & 1 & 0 \\
		1 & 0 & 1 \\
	};

	\matrix (ret) [right=0.2em of K,matrix of nodes,row sep=-\pgflinewidth, nodes={draw}]
	{
		1 & 4 & 3 & |[fill=green!30]| 4 & 1\\
		1 & 2 & 4 & 3 & 3\\
		1 & 2 & 3 & 4 & 1\\
		1 & 3 & 3 & 1 & 1\\
		3 & 3 & 1 & 1 & 0\\
	};

	\draw[very thick, green] (ret-1-4.north west) rectangle (ret-1-4.south east);

	\draw[densely dotted, blue, thick] (mtr-1-4.north west) -- (K-1-1.north west);
	\draw[densely dotted, blue, thick] (mtr-3-4.south west) -- (K-3-1.south west);
	\draw[densely dotted, blue, thick] (mtr-1-6.north east) -- (K-1-3.north east);
	\draw[densely dotted, blue, thick] (mtr-3-6.south east) -- (K-3-3.south east);

	\draw[densely dotted, green, thick] (ret-1-4.north west) -- (K-1-1.north west);
	\draw[densely dotted, green, thick] (ret-1-4.south west) -- (K-3-1.south west);
	\draw[densely dotted, green, thick] (ret-1-4.north east) -- (K-1-3.north east);
	\draw[densely dotted, green, thick] (ret-1-4.south east) -- (K-3-3.south east);

	\matrix (K) [right=0.2em of mtr,matrix of nodes,row sep=-\pgflinewidth, nodes={draw, fill=blue!10}]
	{
		1 & 0 & 1 \\
		0 & 1 & 0 \\
		1 & 0 & 1 \\
	};

	\draw[very thick, blue] (K-1-1.north west) rectangle (K-3-3.south east);
\end{tikzpicture}
\caption{Операция свертки} \label{convolution}
\end{figure}    
        \end{column}
        \begin{column}{.5\paperwidth}
            \begin{table}[H]
    \centering
    \caption{Результаты тестирования моделей} \label{conv-test}
    \begin{tabular}{|c|c|c|c|}
      \hline    
      Сеть         & Кол-во параметров   & Top-1     & Top-5     \\
      \hline
      VGG-16       & 138 357 544         & 71.3\%	& 90.1\%    \\
      \hline
      VGG-19       & 143 667 240         & 71.3\%	& 90.0\%    \\
      \hline
      Inception V3 & 23 851 784          & 77.9\%	& 93.7\%    \\
      \hline
      % ResNet-50    & 25 636 712	         & 74.9\%	& 92.1\%    \\
      % \hline
      % ResNet-101   & 44 707 176	         & 76.4\%	& 92.8\%    \\
      % \hline
      % ResNet-152   & 60 419 944		       & 76.6\%	& 93.1\%    \\
      % \hline
      ResNet-50 V2 & 25 613 800          & 76.0\%	& 93.0\%    \\
      \hline
      ResNet-101 V2 & 44 675 560	       & 77.2\%	& 93.8\%    \\
      \hline
      ResNet-152 V2 & 60 380 648	       & 78.0\%	& 94.2\%    \\
      \hline
      DenseNet-121 & 8 062 504           & 75.0\%	& 92.3\%    \\
      \hline
      DenseNet-169 & 14 307 880          & 76.2\%	& 93.2\%    \\
      \hline
      DenseNet-201 & 20 242 984          & 77.3\%	& 93.6\%    \\
      \hline
    \end{tabular}
  \end{table}
        \end{column}
    \end{columns}   
\end{frame}

\begin{frame}{Оценка качества}
    Для оценки качества работы алгоритмов использовались следующие метрики:
    \begin{itemize}
        \item Precision (Точность): \[P = \frac{TP}{TP+FP}\]
        \item Recall (Полнота): \[R=\frac{TP}{TP+FN}\] 
        \item F1-мера: \[F1=2 \cdot \frac{P \cdot R}{P+R}\] 
        % \item Accuracy: \[A=\frac{TP+TN}{TP+TN+FP+FN}\] 
    \end{itemize}    
    где TP - количество истинно-положительных, TN - истинно-отрицательных, 
    % FP - ложно-положительных, 
    FN - ложно-отрицательных ответов.
\end{frame}

\begin{frame}{Функции потерь}
    \begin{table}[H]
  \centering
  \caption{Популярные функции потерь} \label{loss_funcs}
  \begin{tabular}{|c|c|}
    \hline    
    \hyperlink{name}{Название} & \hyperlink{func}{Функция}\\
    \hline
    Средняя квадратическая ошибка & $E=\frac{1}{N}\displaystyle\sum\limits_{i=1}^{n}(y_i - x_i)^2$\\
    \hline
    Средняя абсолютная ошибка & $E=\frac{1}{N}\displaystyle\sum\limits_{i=1}^{n}|y_i - x_i|$ \\
    \hline
    Верхняя граница & $E=\frac{1}{N}\displaystyle\sum\limits_{i=1}^{n}\max(1-x_i, y_i, 0)$ \\
    \hline
    Перекрестная энтропия & $E=-\displaystyle\sum\limits_{i=1}^{n}(x_i*log(y_i))$ \\
    \hline
  \end{tabular}
\end{table}
    \begin{columns}[T]
        \begin{column}{.8\paperwidth}
            $y_i$ – ожидаемое значение i-го нейрона, $x_i$ – полученное значение i-го нейрона, n – количество выходных нейронов.
        \end{column}
    \end{columns}   
\end{frame}

\begin{frame}{Задача диагностики COVID-19}
    \addthreeimghere{xray-normal}{Норма}{xray-pneumonia}{Пневмония}{xray-covid}{COVID-19}{Рентгеновские снимки грудных клеток}{xrays}
    \begin{columns}[T]
        \begin{column}{.8\paperwidth}            
            Всего было собрано \textbf{14 197} cнимков, из них \textbf{8 066} здоровых пациентов, \textbf{5 558} с пневмонией и \textbf{573} с COVID-19. По 100 изображений каждого класса было отобрано для валидации обучения.    
        \end{column}
    \end{columns}  
\end{frame}

\begin{frame}{Предварительная обработка изображений}
    Способы предварительной обработки:
    \begin{itemize}    
        \item Масштабирование - приведение всех значений в изображении к диапазону [0,1]
        \item Центрирование среднего значения изображения в 0 и нормализация среднеквадратичного отклонения к 1
    \end{itemize}    
    Модели:
    \begin{itemize}
        \item \textbf{Inception V3}, размерность входного слоя: 299x299
        \item \textbf{ResNet-50}, размерность входного слоя: 224x224
        \item \textbf{DenseNet-201}, размерность входного слоя:  224x224
    \end{itemize} 
    Обучение всех моделей проходило по 10 эпох, размер одного пакета - 16 изображений. В качестве функции потерь использовалась категориальная перекрестная энтропия, в качестве оптимизатора - Адам. 
\end{frame}

\begin{frame}{Предварительная обработка изображений}
    \begin{table}[H]
    \centering
    \caption{Результаты обучение моделей со стандартными параметрами \\и предварительным маштабированием изображений} \label{rescale}
    \begin{tabular}{|c|c|c|c|c|c|c|c|}
        \hline    
        Сеть          & loss   & precision & recall & val\_loss & val\_precision & val\_recall \\ % & last\_epoch  \\
        \hline
        Inception V3  & 0.3387 & 0.8860    & 0.8697 & 0.5269    & 0.7204         & 0.6700 \\ % & 10 \\
        \hline
        ResNet-50     & 0.3242 & 0.8870    & 0.8732 & 0.2373    & 0.7354         & 0.7133 \\ % & 8  \\
        \hline
        % DenseNet-201  & 0.3655 & 0.8742    & 0.8622 & 0.3539    & 0.7643         & 0.7133 \\ % & 10 \\
        \hline
      \end{tabular}
\end{table}

\begin{table}[H]
    \centering
    \caption{Результаты обучение моделей со стандартными параметрами \\и предварительным центрированием изображений} \label{samplewise}    
    \begin{tabular}{|c|c|c|c|c|c|c|c|}
        \hline    
        Сеть          & loss   & precision & recall & val\_loss & val\_precision & val\_recall \\ % & epoch  \\
        \hline
        Inception V3  & 0.2352 & 0.9176    & 0.9106 & 0.2737   & 0.7884        & 0.7700 \\ % & 10 \\
        \hline
        ResNet-50     & 0.3353  & 0.8848   & 0.8705 & 0.9311   & 0.6537        & 0.6167 \\ % & 8  \\
        \hline
        DenseNet-201  & 0.3655  & 0.8742   & 0.8622 & 0.3539   & 0.7643        & 0.7133 \\ % & 10 \\
        \hline
      \end{tabular}
\end{table}
\end{frame}

\begin{frame}{Выбор оптимизатора}
    \begin{columns}[T]
        \begin{column}{.3\paperwidth}
            Параметры:
            \begin{itemize}
                \item количество эпох: 10
                \item размер пакета: 8 
                \item Функция потерь: категориальная перекрестная энтропия 
                \item размер входов: 500x500
            \end{itemize}
        \end{column}
        \begin{column}{.2\paperwidth}
            Модели:            
            \begin{itemize}
                \item Inception V3
                \item ResNet-50 V2
                \item DenseNet-201
            \end{itemize} 
        \end{column}
        \begin{column}{.3\paperwidth}
            Оптимизаторы:
            \begin{itemize}
                \item SGD - Стохастический градиентный спуск
                \item RMSprop
                \item Adam
            \end{itemize} 
        \end{column}
    \end{columns}    
\end{frame}

\begin{frame}{Результаты}
    
    % \begin{columns}[T]
    %     \begin{column}{.2\paperwidth}
            \begin{itemize}
                \item Adam:   \begin{table}[H]
    \centering
    % \caption{Результаты обучения сетей с оптимизатором Adam} \label{adam}
    \resizebox{0.7\textwidth}{!}{
        \begin{tabular}{|c|c|c|c|c|c|c|c|c|c|}
            \hline  
             \multirow{2}{*}{} & \multicolumn{3}{c|}{Inception V3}   & \multicolumn{3}{c|}{ResNet-50 V2}    & \multicolumn{3}{c|}{DenseNet-201}   \\ 
             \cline{2-10} 
                            & precision	    &	recall	&	f1-score &	precision  &	recall	&	f1-score &	precision	&	recall  & f1-score \\
            \hline							
            COVID-19	    &	0.53	    &	0.38	&	0.44	 &	0.55	   &	0.40    &	0.46     &	0.52	&	0.56	&	0.54    \\
            \hline							
            Normal	        &	0.51	    &	0.58	&	0.54	 &	0.51	   &	0.53	&	0.52     &	0.53	&	0.51	&	0.52   \\
            \hline							
            Pneumonia	    &	0.59	    &	0.68	&	0.63	 &	0.52	   &	0.63	&	0.57     &	0.53	&	0.51	&	0.52  \\
            \hline						
          \end{tabular}
    }
\end{table}
                \item RMSprop: \begin{table}[H]
    \centering
    \caption{Результаты обучения сетей с оптимизатором RMSprop} \label{rms}
    \resizebox{1\textwidth}{!}{
        \begin{tabular}{|c|c|c|c|c|c|c|c|c|c|}
            \hline  
             \multirow{2}{*}{} & \multicolumn{3}{c|}{Inception V3}   & \multicolumn{3}{c|}{ResNet-50 V2}    & \multicolumn{3}{c|}{DenseNet-201}   \\ 
             \cline{2-10} 
                            & precision	    &	recall	&	f1-score &	precision  &	recall	&	f1-score &	precision	&	recall  & f1-score \\
            \hline							
            COVID-19	    &	0.57	&	0.51	&	0.54	 &	0.56	&	0.84	&	0.67    &	0.53	&	0.23	&	0.32    \\
            \hline							
            Normal	       	&	0.52	&	0.70	&	0.60	 &	0.63	&	0.36	&	0.46    &	0.56	&	0.65	&	0.60	\\
            \hline							
            Pneumonia	    &	0.53	&	0.58	&	0.55	 &	0.54	&	0.49	&	0.51	&	0.55	&	0.77	&	0.64	\\   
            \hline						
          \end{tabular}
    }
\end{table}
                \item SGD:    \begin{table}[H]
    \centering
    % \caption{Результаты обучения сетей с оптимизатором SGD} \label{sgd}
    \resizebox{0.7\textwidth}{!}{
        \begin{tabular}{|c|c|c|c|c|c|c|c|c|c|}
            \hline  
             \multirow{2}{*}{} & \multicolumn{3}{c|}{\textbf{Inception V3}}   & \multicolumn{3}{c|}{ResNet-50 V2}    & \multicolumn{3}{c|}{DenseNet-201}   \\ 
             \cline{2-10} 
                            & \textbf{precision}	    &	\textbf{recall}	&	\textbf{f1-score} &	precision  &	recall	&	f1-score &	precision	&	recall  & f1-score \\
            \hline							
            COVID-19	    &	\textbf{0.63}	    &	\textbf{0.50}    &	\textbf{0.56}	 &	0.50	   &	0.58	&	0.54     &	0.33	&	0.88	&	0.48    \\
            \hline							
            Normal	       	&	\textbf{0.60}        &	\textbf{0.64}	&	\textbf{0.62}	 &	0.56	   &	0.55	&	0.55     &	0	&	0	&	0	\\
            \hline							
            Pneumonia	    &	\textbf{0.67}	    &	\textbf{0.64}	&	\textbf{0.65}	 &	0.52	   &	0.45	&	0.48     &	0.33	&	0.11	&	0.17	\\   
            \hline						
          \end{tabular}
    }
\end{table}

    
            \end{itemize}
        % \end{column}
        % \begin{column}{.8\paperwidth}
            
          
           
    %     \end{column}
    % \end{columns}   
\end{frame}

% \begin{frame}{Обучение сетей}
%     \begin{columns}[T]
%         \begin{column}{.4\paperwidth}
%             \begin{figure}
%                 \centering
%                 \includegraphics[width=\textwidth]{train_accuracy.pdf} 
%                 \caption{Обучение}
%             \end{figure}            
%         \end{column}
%         \begin{column}{.4\paperwidth}
%             \begin{figure}
%                 \centering                
%                 \includegraphics[width=\textwidth]{val_accuracy.pdf}
%                 \caption{Тестирование}
%             \end{figure} 
%         \end{column}
%     \end{columns}   
% \end{frame}
\begin{frame}{Заключение}
    \textbf{Итоги}:\\
    \begin{itemize}
        \item Проведено исследование применения глубокого обучения в задачах распознавания изображений.
        \item Изучен вопрос диагностики COVID-19 и пневмонии по рентгеновским снимкам грудной клетки.
        \item Проведен анализ вариантов предварительной обработки изображений для решения данной задачи. 
        \item Разработаны и обучены модели Inception, ResNet и DenseNet для диагностики COVID-19.
        \item Проведен анализ результатов тестирования по метрикам: точность, полнота и F1.
        \item Выявлено, что для решения задачи диагностики COVID-19 предпочтительно использовать сеть \textbf{Inception} с оптимизатором \textbf{SGD}.
    \end{itemize}
\end{frame}
\begin{frame}
    \centering\Huge
    Спасибо за внимание!
\end{frame}
\end{document}