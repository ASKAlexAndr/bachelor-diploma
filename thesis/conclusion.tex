\anonsection{Заключение}
В данной работе было проведено исследование алгоритмов глубокого машинного обучения в задачах распознавания изображений. Были рассмотрены, обучены и протестированы различные архитектуры сверточных нейронных сетей на задаче диагностирования пневмонии и COVID-19 по рентгеновским снимкам грудной клетки. 

Результаты исследования показывают, что глубокие нейронные сети эффективно справляются с распознаванием изображений. Самую высокую точность показала плотная сверточная нейронная сеть - DenseNet с точностью 88.54\%.
Расширение обучающих данных, увеличение параметров моделей и большее количество эпох может значительно повысить качество распознавания изображений.


\clearpage