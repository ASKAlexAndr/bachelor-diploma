% !TEX program = xelatex

% Author: Alexandr Korotkov
% https://github.com/ASKAlexAndr/bachelor-thesis

\documentclass[xetex, a4paper,12pt]{extarticle} 
% \usepackage[utf8]{inputenc}
\usepackage[russian]{babel}
% \usepackage{csquotes}

\usepackage{fontspec}
% \usepackage{xunicode}
% \usepackage{pdfpages}
% \usepackage{xltxtra}

% \defaultfontfeatures{Ligatures=TeX}
\setmainfont{Times New Roman} % Нормоконтроллеры хотят именно его
\usefonttheme{serif}

\usepackage{tikz} % Для создания рисунков с помощью tikz
\usepackage{listings} % Для листингов программ

% \usecolortheme{whale}
\usecolortheme{dolphin} 
% \usetheme{warsaw}
%\usetheme{Madrid}

\useinnertheme{rounded} % Скругляем углы блоков
\useoutertheme{infolines} % Добавляем колонтитулы
\setbeamertemplate{headline}{} % Убираем верхний колонтитул
\setbeamertemplate{navigation symbols}{} % Убираем навигационные символы
\usefonttheme[onlysmall]{structurebold} % жирный шрифт в колонтитулах
%\setbeamerfont*{frametitle}{size=\normalsize,series=\bfseries}
\setbeamercovered{transparent} % uncover будет рисовать закрытые части текста серым
%\setbeamercovered{invisible} % а так -- невидимым
% Определяем цвет структурных элементов
\definecolor{unicover}{rgb}{0.10,0.35,0.62} 
\setbeamercolor{structure}{fg=unicover}
% Добавляем логотип университета
\pgfdeclareimage[height=1.5cm]{university-logo}{LogoYarSU.png}
\logo{\pgfuseimage{university-logo}}

% Делим нижний колонтитул на три части: автор, название, число слайдов
\newcommand{\makefootline}[3]{
	\setbeamertemplate{footline}{
		\leavevmode%
		\hbox{%
			\begin{beamercolorbox}[wd=#1\paperwidth,ht=2.25ex,dp=1ex,center]{author in head/foot}
				\usebeamerfont{author in head/foot}%
				\insertshortauthor
			\end{beamercolorbox}
			\begin{beamercolorbox}[wd=#2\paperwidth,ht=2.25ex,dp=1ex,center]{title in head/foot}
				\usebeamerfont{title in head/foot}\insertshorttitle
			\end{beamercolorbox}
			\begin{beamercolorbox}[wd=#3\paperwidth,ht=2.25ex,dp=1ex,right]{date in head/foot}
				\insertframenumber{} / \inserttotalframenumber\hspace*{2ex}
			\end{beamercolorbox}
		}%
	}
}
\makefootline{.2}{.7}{.1} % Сумма длин = 1

% Делаем блоки прозрачными, чтобы они не закрывали логотип
\setbeamertemplate{blocks}[default]
\setbeamercolor{block title}{bg=}
\setbeamercolor{block body}{bg=}

 

\begin{document}

\centering{\textit{\textbf{Аннотация \\ к выпускной квалификационной работе бакалавра}}}

\centering{по направлению 01.03.02 «Прикладная математика и информатика»}

\vspace{2em}
\centering{\textbf{Анализ алгоритмов глубокого машинного обучения \\ в задачах распознавания изображений}}
\vspace{2em}

\begin{flushleft}
\textbf{
    Студента группы ИВТ-41 Короткова Александра Сергеевича\\
    Кафедра дискретного анализа\\
    Научный руководитель - Матвеев Д.В. ст. преподаватель, к.т.н.
}
\vspace{2em}
    
\textbf{Актуальность:} глубокое машинное обучение широко применяется в различных сферах человеческой деятельности. 
\vspace{1em}

\textbf{Цель:} Изучить и проанализировать применение алгоритмов глубокого машинного обучения в задачах распознавания изображений. \\

\textbf{Задачи: }
\begin{itemize}
    \item Обзор существующих моделей глубоких нейронных сетей и их применение в классификации изображений.
    \item Изучение вопроса диагностики COVID-19 по рентгеновским снимкам грудной клетки. 
    \item Разработка и адаптация алгоритмов для решение поставленной задачи.
    \item Оценка эффективности разработанных моделей.
    \item Анализ полученных результатов.
\end{itemize}
\vspace{1em}

\textbf{Содержательная часть:} в работе рассмотрены различные архитектуры сверточных нейронных сетей (VGG, Inception, ResNet, DenseNet). Проведен анализ вариантов предварительной обработки изображений и сравнение качества различных алгоритмов оптимизации для данных сетей. 
\vspace{1em}

\textbf{Результаты:} Обучение и тестирование моделей, анализ полученных результатов и выявление самого эффективного алгоритма.
\vspace{1em}

\end{flushleft}

\end{document}