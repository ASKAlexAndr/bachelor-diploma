\section{Программная реализация}

\subsection{Конфигурация системы}
Все модели были протестированы на системе со следующими характеристиками:
\begin{itemize}
  \item Операционная система - Ubuntu 19.10
  \item Процессор - Intel Core i5 9400F CPU 2.90 Ghz
  \item Объем оперативной памяти - 8 Гб
  \item Видеокарта - Nvidia GeForce GTX 1660
\end{itemize}

\subsection{Средства реализации}
Python 3 - гибкий и мощный язык программирования, эффективно выполняющий задачи анализа и обработки данных. 

TensorFlow - многофункциональный фреймворк с открытым исходным кодом, разработанный компанией Google для решения задач машинного обучения. TensorFlow позволяет проектировать и обучать различные архитектуры нейронных сетей.

cuDNN - библиотека глубоких нейронных сетей от Nvidia позволяющая использовать для вычеслений мощности графического процессора. 
\subsection{Обучающая выборка}
%%%
Создание обучающей выборки в задачах такого типа является сложным
и долгим процессом. Необходимо большое количество времени и помощь
большого количества людей для того, чтобы создать и разметить
даже небольшую выборку.
%%%
Поэтому, в качестве источника обучающих и тестовых данных была использована готовая база данных: ???

\subsection{X1}

\subsection{X2}

\subsection{X3}


\clearpage