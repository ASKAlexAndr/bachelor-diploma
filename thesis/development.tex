\section{Анализ алгоритмов} \label{Dev}

\subsection{Постановка задачи}
COVID-19 - тяжелая респираторная инфекция, вызываемая коронавирусом SARS-CoV-2. На момент написания работы, вспышка вируса, возникщая в китайском городе Ухань, переросла в глобальную пандемию. Решение задачи автоматической диагностики данного заболевания позволит снизить нагрузку на врачей и повысит эффективность их работы.

Пневмония, которую порождает COVID-19, чаще всего является двусторонней и имеет периферическую локализацию, что дает возможность визуально отличить её частые случаи от других видов пневмоний.

\addthreeimghere{xray-normal}{Норма}{xray-pneumonia}{Пневмония}{xray-covid}{COVID-19}{Рентгеновские снимки грудных клеток}{xrays}

\subsection{Средства реализации}
cuDNN - библиотека глубоких нейронных сетей от Nvidia позволяющая использовать для вычеслений мощности графического процессора.

Python 3 - гибкий и мощный язык программирования, эффективно выполняющий задачи анализа и обработки данных.

TensorFlow - многофункциональный фреймворк с открытым исходным кодом, разработанный компанией Google, позволяющий проектировать и обучать различные архитектуры нейронных сетей.

Keras - высокоуровневый API для решения задач глубокого машинного обучения, входящий в состав TensorFlow.

% \subsection{Конфигурация системы}
% Все модели были протестированы на системе со следующими характеристиками:
% \begin{itemize}
%   \item Операционная система - Ubuntu 20.04
%   \item Процессор - Intel Core i5 9400F CPU 2.90 Ghz
%   \item Объем оперативной памяти - 8 Гб
%   \item Видеокарта - Nvidia GeForce GTX 1660
% \end{itemize}

\subsection{Обучающая выборка}
Сбор данных для обучения нейронных сетей в задачах такого типа - сложный и долгий процесс, который требует затрат большого количества времени и участие большого количества людей. Поэтому, в качестве источника обучающих и тестовых данных использовались готовые, уже размеченные датасеты: \cite{tawsifurrahman}, \cite{cohen2020covid} и \cite{wang2020covidnet}. В общем случае было получено 13897 тренировочных и 1579 проверочных изображений.

Снимки из обучающей выборки имеют разное разрешение, однако нейронные сети требуют заранее установленное количество входных нейронов. Поэтому, в качестве предварительной обработки данных, все изображения перед подачей на сеть маштабируются к одному разрешению: 512x512 px.

\subsection{Эксперименты и результаты}
В ходе экспериментов были модернизированы для задачи, обучены и протестированы следующие модели нейронных сетей: Inception-V3, ResNet-50 и DenseNet-201. Все модели обучались на общем наборе тренировочных и проверочных данных и с одинаковыми параметрами:
\begin{itemize}
    \item Размер одного пакета - 8 изображений
    \item Количество эпох - 10
\end{itemize}
Точность сетей в процессе обучения и тестирования показана на рис. \ref{train-results}. Финальные результаты экспериментов указаны в таблице \ref{test-results}

\addtwoimghere{train_accuracy}{Обучение}{val_accuracy}{Тестирование}{Точность сетей}{train-results}

\begin{table}[H]
    \centering
    \caption{Результаты экспериментов} \label{test-results}
    \begin{tabular}{|c|c|c|}
      \hline    
      \hyperlink{network}{Сеть} & \hyperlink{accuracy}{Точность при обучении} & \hyperlink{val_accuracy}{Точность на проверочных данных}\\
      \hline
      Inception & 89.93\% & 87.90\% \\
      \hline
      ResNet & 88.87\% & 73.65\% \\
      \hline
      DenseNet & 89.86\% & 88.54\% \\
      \hline
    \end{tabular}
  \end{table}

\clearpage