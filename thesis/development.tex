\section{Анализ алгоритмов} \label{Dev}

\subsection{Постановка задачи}
COVID-19 - тяжелая респираторная инфекция, вызываемая коронавирусом SARS-CoV-2. На момент написания работы, вспышка вируса, возникшая в китайском городе Ухань, переросла в глобальную пандемию. Основным способом диагностики COVID-19 является метод полимеразной цепной реакции (ПЦР) с обратной транскрипцией, однако это трудоемкий, требующий участия человека процесс. Альтернативой является рентгенологической исследование, например рентген грудной клетки или компьютерная томография (КТ). Исследования показывают, что большинство снимков пациентов с COVID-19 содержат специфические аномалии, такие как двустороннее ретикулярное узловое затемнение или синдром матового стекла, что позволяет визуально отличить заболевание от других видов пневмоний \cite{wang2020covidnet}. Несмотря на то, что рентген обладает меньшей чувствительностью чем КТ или ПЦР, это гораздо более доступный и быстрый метод диагностики, что является существенным критерием в период пандемии. Решение задачи автоматической диагностики данного заболевания позволит снизить нагрузку на врачей и повысит эффективность их работы.

\addthreeimghere{xray-normal}{Норма}{xray-pneumonia}{Пневмония}{xray-covid}{COVID-19}{Рентгеновские снимки грудных клеток}{xrays}

\subsection{Средства реализации}
% cuDNN - библиотека глубоких нейронных сетей от Nvidia позволяющая использовать для вычеслений мощности графического процессора.
Python 3 - гибкий и мощный язык программирования, эффективно выполняющий задачи анализа и обработки данных.

TensorFlow - многофункциональный фреймворк с открытым исходным кодом, разработанный компанией Google, позволяющий проектировать и обучать различные архитектуры нейронных сетей.

Keras - высокоуровневый API для решения задач глубокого машинного обучения, входящий в состав TensorFlow.

% \subsection{Конфигурация системы}
% Все модели были протестированы на системе со следующими характеристиками:
% \begin{itemize}
%   \item Операционная система - Ubuntu 20.04
%   \item Процессор - Intel Core i5 9400F CPU 2.90 Ghz
%   \item Объем оперативной памяти - 8 Гб
%   \item Видеокарта - Nvidia GeForce GTX 1660
% \end{itemize}

\subsection{Оценка качества}
Для оценки качества работы алгоритмов использовались следующие метрики:
\begin{itemize}
    \item Precision (Точность): \[P = \frac{TP}{TP+FP}\]
    \item Recall (Полнота): \[R=\frac{TP}{TP+FN}\] 
    \item F1-мера: \[F1=2 \cdot \frac{P \cdot R}{P+R}\] 
    % \item Accuracy: \[A=\frac{TP+TN}{TP+TN+FP+FN}\] 
\end{itemize}
где TP - количество истинно-положительных, TN - истинно-отрицательных, 
% FP - ложно-положительных, 
FN - ложно-отрицательных ответов.\\
Precision обозначает долю правильно идентифицированных объектов класса относительно всех объектов причисленных к этому классу. \\Recall показывет долю найденных сетью элементов класса относительно всех элементов этого класса.\\ 
F1 объединяет precision и recall вычисляя их гармоническое среднее.
Также в процессе обучения сети в качестве характеристики качества обучения рассматривались значения функции потерь.

\subsection{Обучающая выборка}
Сбор данных для обучения нейронных сетей в задачах такого типа - сложный и долгий процесс, который требует затрат большого количества времени и участие большого количества людей. Поэтому, в качестве источника обучающих и проверочных данных использовались готовые, уже размеченные датасеты: \cite{tawsifurrahman}, \cite{cohen2020covid} и \cite{wang2020covidnet}. 
Всего было собрано 14 197 cнимков, из них 8 066 здоровых пациентов, 5 558 с пневмонией и 573 с COVID-19. По 100 изображений каждого класса было отобрано для валидации обучения.

% Общее количество снимков указано в таблице \ref{inputs}

% \begin{table}[H]
    \centering
    \caption{Количество входных данных} \label{inputs}
    \begin{tabular}{|c|c|c|}
      \hline    
                & Обучение  & Проверка  \\
      \hline
      Норма     & 7 966     & 885       \\
      \hline
      Пневмония & 5 458     & 594       \\
      \hline
      COVID-19  & 473       & 100       \\
      \hline
      Итого     & 13 897    & 1 579     \\
      \hline
    \end{tabular}
  \end{table}

% Снимки из обучающей выборки имеют разное разрешение, однако нейронные сети требуют заранее установленное количество входных нейронов. Поэтому, в качестве предварительной обработки данных, все изображения перед подачей на сеть масштабируются к одному разрешению: 512x512 px.
% \clearpage
\subsection{Предварительная обработка изображений}

При использовании глубоких нейронных сетей в задачах классификации изображений требуется проведение дополнительных исследований, с целью выбора оптимальных параметров различных частей алгоритмов. 

Предварительная обработка входных данных может существенно повлиять на обучение моделей. В рамках данной части исследования рассматривались следующие варианты обработки изображений:
\begin{itemize}    
    \item Масштабирование - деление всех значений в изображении на 255.
    \item Центрирование среднего значения изображения в 0 и нормализация среднеквадратичного отклонения в 1.
\end{itemize}
В исследовании были протестированы следующие модели нейронных сетей со стандартными параметрами:
\begin{itemize}
    \item Inception V3, размерность входного слоя: 299x299
    \item ResNet-50, размерность входного слоя: 224x224
    \item DenseNet-201, размерность входного слоя:  224x224
\end{itemize} 
Обучение всех моделей проходило по 10 эпох, размер одного пакета - 16 изображений. В качестве функции потерь использовалась категориальная перекрестная энтропия, в качестве оптимизатора - Адам. 
% Обучение останавливалось заранее, если значение функции loss на валидационной выборке не уменьшалось в течение пяти эпох. 
Значения функции ошибки (loss), метрик precision и recall  по итогам обучения и валидации (приставка «val\_») для обработанных изображений методами масштабирования и центрирования указаны в таблицах \ref{rescale} и \ref{samplewise} соответственно. 

\begin{table}[H]
    \centering
    \caption{Результаты обучение моделей со стандартными параметрами \\и предварительным масштабированием изображений} \label{rescale}
    \begin{tabular}{|c|c|c|c|c|c|c|c|}
        \hline    
        Сеть          & loss   & precision & recall & val\_loss & val\_precision & val\_recall \\ % & last\_epoch  \\
        \hline
        Inception V3  & 0.2352 & 0.9176    & 0.9106 & 0.2737   & 0.7884        & 0.7700 \\ % & 10 \\
        \hline
        ResNet-50     & 0.3242 & 0.8870    & 0.8732 & 0.2373    & 0.7354         & 0.7133 \\ % & 8  \\
        \hline
        DenseNet-201  & 0.2742 & 0.9054    & 0.8969 & 0.3196    & 0.7560         & 0.7333 \\ % & 10 \\
        \hline
      \end{tabular}
\end{table}

\begin{table}[H]
    \centering
    \caption{Результаты обучение моделей со стандартными параметрами \\и предварительным центрированием изображений} \label{samplewise}    
    \begin{tabular}{|c|c|c|c|c|c|c|c|}
        \hline    
        Сеть          & loss   & precision & recall & val\_loss & val\_precision & val\_recall \\ % & epoch  \\
        \hline
        Inception V3  & 0.3387 & 0.8860    & 0.8697 & 0.5269    & 0.7204         & 0.6700 \\ % & 10 \\
        \hline
        ResNet-50     & 0.3353  & 0.8848   & 0.8705 & 0.9311   & 0.6537        & 0.6167 \\ % & 8  \\
        \hline
        DenseNet-201  & 0.3655  & 0.8742   & 0.8622 & 0.3539   & 0.7643        & 0.7133 \\ % & 10 \\
        \hline
      \end{tabular}
\end{table}


% Confusion Matrix
% [[12 43 45]
%  [10 57 33]
%  [14 51 35]]
% Classification Report
%               precision    recall  f1-score   support

%       normal       0.33      0.12      0.18       100
%    pneumonia       0.38      0.57      0.45       100
%     COVID-19       0.31      0.35      0.33       100

%     accuracy                           0.35       300
%    macro avg       0.34      0.35      0.32       300
% weighted avg       0.34      0.35      0.32       300

Как видно из таблиц предварительное масштабирование значений дает значения метрик лучше чем центрирование в процессе обучения и валидации. При этом лучшие результаты в данном тестировании показала сеть Inception V3.

\subsection{Выбор оптимизатора}
На качество работы нейронных сетей может существенно влиять выбранный алгоритм оптимизации функции ошибки. В данной части исследования проводится анализ различных оптимизаторов: Стохастический градиентный спуск (SGD), RMSProp и Adam. Тестирование проводилось на сетях Inception V3, ResNet-50, DenseNet-201 и предварительно масштабированными входными изображениями размером 500x500 пикселей.  

Все модели обучались по 10 эпох, размер одного пакета ­- 8 изображений. В таблицах \ref{adam}, \ref{rms} и \ref{sgd} указаны полученные в итоге при валидации значения метрик Precision, Recall и F1 для каждого класса.

\begin{table}[H]
    \centering
    \caption{Результаты обучения сетей с оптимизатором Adam} \label{adam}
    \resizebox{1\textwidth}{!}{
        \begin{tabular}{|c|c|c|c|c|c|c|c|c|c|}
            \hline  
             \multirow{2}{*}{} & \multicolumn{3}{c|}{Inception V3}   & \multicolumn{3}{c|}{ResNet-50 V2}    & \multicolumn{3}{c|}{DenseNet-201}   \\ 
             \cline{2-10} 
                            & precision	    &	recall	&	f1-score &	precision  &	recall	&	f1-score &	precision	&	recall  & f1-score \\
            \hline							
            COVID-19	    &	0.53	    &	0.38	&	0.44	 &	0.55	   &	0.40    &	0.46     &	0.52	&	0.56	&	0.54    \\
            \hline							
            Normal	        &	0.51	    &	0.58	&	0.54	 &	0.51	   &	0.53	&	0.52     &	0.53	&	0.51	&	0.52   \\
            \hline							
            Pneumonia	    &	0.59	    &	0.68	&	0.63	 &	0.52	   &	0.63	&	0.57     &	0.53	&	0.51	&	0.52  \\
            \hline						
          \end{tabular}
    }
\end{table}
\begin{table}[H]
    \centering
    \caption{Итоги обучения сетей с оптимизатором RMSprop} \label{rms}
    \resizebox{1\textwidth}{!}{
        \begin{tabular}{|c|c|c|c|c|c|c|c|c|c|}
            \hline  
             \multirow{2}{*}{} & \multicolumn{3}{c|}{Inception V3}   & \multicolumn{3}{c|}{ResNet-50 V2}    & \multicolumn{3}{c|}{DenseNet-201}   \\ 
             \cline{2-10} 
                            & precision	    &	recall	&	f1-score &	precision  &	recall	&	f1-score &	precision	&	recall  & f1-score \\
            \hline							
            COVID-19	    &	0.57	&	0.51	&	0.54	 &	0.56	&	0.84	&	0.67    &	0.53	&	0.23	&	0.32    \\
            \hline							
            Normal	       	&	0.52	&	0.70	&	0.60	 &	0.63	&	0.36	&	0.46    &	0.56	&	0.65	&	0.60	\\
            \hline							
            Pneumonia	    &	0.53	&	0.58	&	0.55	 &	0.54	&	0.49	&	0.51	&	0.55	&	0.77	&	0.64	\\   
            \hline						
          \end{tabular}
    }
\end{table}
\begin{table}[H]
    \centering
    \caption{Результаты обучения сетей с оптимизатором SGD} \label{sgd}
    \resizebox{1\textwidth}{!}{
        \begin{tabular}{|c|c|c|c|c|c|c|c|c|c|}
            \hline  
             \multirow{2}{*}{} & \multicolumn{3}{c|}{Inception V3}   & \multicolumn{3}{c|}{ResNet-50 V2}    & \multicolumn{3}{c|}{DenseNet-201}   \\ 
             \cline{2-10} 
                            & precision	    &	recall	&	f1-score &	precision  &	recall	&	f1-score &	precision	&	recall  & f1-score \\
            \hline							
            COVID-19	    &	0.63	    &	0.50    &	0.56	 &	0.50	   &	0.58	&	0.54     &	0.33	&	0.88	&	0.48    \\
            \hline							
            Normal	       	&	0.60        &	0.64	&	0.62	 &	0.56	   &	0.55	&	0.55     &	0	&	0	&	0	\\
            \hline							
            Pneumonia	    &	0.67	    &	0.64	&	0.65	 &	0.52	   &	0.45	&	0.48     &	0.33	&	0.11	&	0.17	\\   
            \hline						
          \end{tabular}
    }
\end{table}



\subsection{Результаты}
Таблицы \ref{adam}, \ref{rms} и \ref{sgd} показывают, что различные оптимизаторы по разному эффективны для различных сетей, так сеть ResNet-50 V2 вместе с алгоритмом оптимизации RMSProp по метрике Recall смогла идентифицировать большее количество изображений с COVID-19, при этом Inception V3 с методом SGD, в среднем по классам показывает самые высокие результаты, согласно метрике F1. 

Таким образом, в ходе проведенного исследования было выявлено, что для решения задачи диагностики COVID-19 по рентгеновским снимкам предпочтительно использовать сеть Inception в совокупности с алгоритмом оптимизации SGD. При этом значения во входных изображениях рекомендуется приводить к диапазону [0;1] посредством масштабирования. 

% В ходе экспериментов были модернизированы для задачи, обучены и протестированы  модели на базе следующих архитектур: Inception, ResNet и DenseNet. 
% Точность сетей по метрике accuracy в процессе обучения и тестирования показана на рис. \ref{train-results}. 

% \addtwoimghere{train_accuracy}{Обучение}{val_accuracy}{Тестирование}{Точность сетей}{train-results}

% Как видно из рис. \ref{train-results} точность сетей возрастает с повышением количества эпох обучения и стремится к 1, что говорит о высоком качестве работы данных алгоритмов в решении задач классификации изображений.

\clearpage