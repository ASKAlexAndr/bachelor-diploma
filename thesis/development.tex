\section{Программная реализация}

\subsection{Постановка задачи}

COVID-19 - тяжелая респираторная инфекция, вызываемая коронавирусом SARS-CoV-2. На момент написания работы, вспышка вируса, возникщая в китайском городе Ухань, переросла в глобальную пандемию. Решение задачи автоматической диагностики данного заболевания позволит снизить нагрузку на врачей и повысит эффективность их работы.

Пневмония, которую порождает COVID-19, чаще всего является двусторонней и имеет периферическую локализацию, что позволяет визуально отличить её от стандартной пневмонии.

\addthreeimghere{xray-normal}{Норма}{xray-pneumonia}{Пневмония}{xray-covid}{COVID-19}{Рентгеновские снимки грудных клеток}{xrays}

% \subsection{Конфигурация системы}
% Все модели были протестированы на системе со следующими характеристиками:
% \begin{itemize}
%   \item Операционная система - Ubuntu 20.04
%   \item Процессор - Intel Core i5 9400F CPU 2.90 Ghz
%   \item Объем оперативной памяти - 8 Гб
%   \item Видеокарта - Nvidia GeForce GTX 1660
% \end{itemize}

\subsection{Средства реализации}
Python 3 - гибкий и мощный язык программирования, эффективно выполняющий задачи анализа и обработки данных.

TensorFlow - многофункциональный фреймворк с открытым исходным кодом, разработанный компанией Google, позволяющий проектировать и обучать различные архитектуры нейронных сетей.

Keras - высокоуровневый API для решения задач глубокого машинного обучения, входящий в состав TensorFlow.

% cuDNN - библиотека глубоких нейронных сетей от Nvidia позволяющая использовать для вычеслений мощности графического процессора.

\subsection{Обучающая выборка}
%%%
Создание обучающей выборки в задачах такого типа является сложным
и долгим процессом. Необходимо большое количество времени и помощь
большого количества людей для того, чтобы создать и разметить
даже небольшую выборку.
%%%
Поэтому, в качестве источника обучающих и тестовых данных использовались готовые наборы данных: \cite{tawsifurrahman}, \cite{cohen2020covid}, \cite{wang2020covidnet}


\subsection{Разработка системы}
Снимки из обучающей выборки имеют разное разрешение, однако нейронная сеть требует заранее установленное количество входных нейронов. Поэтому, в качестве предварительной обработки данных, все изображения на входе маштабируются к одному разрешению: 512x512 px.

Пораженные болезнью участки могут занимать различные места на снимках, поэтому для нахождения этих участков хорошо подходит сверточная нейронная сеть.

% \subsection{Сверточная нейронная сеть}
% Первой моделью для реализации была выбрана сверточная нейронная сеть, содержащая:
% \begin{itemize}
%   \item 3 сверточных слоя c активационной функцией ReLU
%   \item 3 пулинговых слоя
%   \item 1 выравнивающий слой, преобразующий двумерную матрицу значений в вектор
%   \item 1 полносвязный слой c активационной функцией ReLU
%   \item 1 полносвязный слой c активационной функцией Softmax
% \end{itemize}

% По итогам обучения точность сети составила 88.01\%
\clearpage