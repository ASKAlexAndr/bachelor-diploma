\section{Программная реализация}

\subsection{Постановка задачи}

COVID-19 - тяжелая респираторная инфекция, вызываемая коронавирусом SARS-CoV-2. На момент написания работы, вспышка вируса, возникщая в китайском городе Ухань, переросла в глобальную пандемию. Решение задачи автоматической диагностики данного заболевания позволит снизить нагрузку на врачей и повысит эффективность их работы.

Пневмония, которую порождает COVID-19, чаще всего является двусторонней и имеет периферическую локализацию, что позволяет визуально отличить её от стандартной пневмонии.

\addthreeimghere{xray-normal}{Норма}{xray-pneumonia}{Пневмония}{xray-covid}{COVID-19}{Рентгеновские снимки грудных клеток}{xrays}

\subsection{Средства реализации}
cuDNN - библиотека глубоких нейронных сетей от Nvidia позволяющая использовать для вычеслений мощности графического процессора.

Python 3 - гибкий и мощный язык программирования, эффективно выполняющий задачи анализа и обработки данных.

TensorFlow - многофункциональный фреймворк с открытым исходным кодом, разработанный компанией Google, позволяющий проектировать и обучать различные архитектуры нейронных сетей.

Keras - высокоуровневый API для решения задач глубокого машинного обучения, входящий в состав TensorFlow.

\subsection{Конфигурация системы}
Все модели были протестированы на системе со следующими характеристиками:
\begin{itemize}
  \item Операционная система - Ubuntu 20.04
  \item Процессор - Intel Core i5 9400F CPU 2.90 Ghz
  \item Объем оперативной памяти - 8 Гб
  \item Видеокарта - Nvidia GeForce GTX 1660
\end{itemize}

\subsection{Обучающая выборка}
Сбор данных для обучения нейронных сетей в задачах такого типа - сложный и долгий процесс, который требует затрат большого количества времени и участие большого количества людей. Поэтому, в качестве источника обучающих и тестовых данных были взяты и сгруппированы готовые датасеты: \cite{tawsifurrahman}, \cite{cohen2020covid}, \cite{wang2020covidnet}

Снимки из обучающей выборки имеют разное разрешение, однако нейронные сети требуют заранее установленное количество входных нейронов. Поэтому, в качестве предварительной обработки данных, все изображения на входе маштабируются к одному разрешению: 512x512 px.

\subsection{Эксперименты}
В ходе работы были разработаны, обучены и протестированы модели: ???.  

\clearpage