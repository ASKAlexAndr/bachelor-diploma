\section{Сверточные нейронные сети} \label{CNN}

\subsection{Архитектура}

Большая часть современных нейронных сетей направленных на анализ изображений базируются на архитектуре сверточной нейронной сети.
Ранние нейронные сети состояли из полносвязных слоев - слоев, в которых каждый нейрон связан с каждым нейроном следующего слоя, что значительно увеличивало вычислительную сложность системы при увеличении количества нейронов. 
В типовых сверточных нейронных сетях преимущественно используются сверточные слои. 

Сверточные слои характеризуются использованием матриц весов, называемых фильтрами или ядрами, которые обладают размерностью меньше исходных данных. Такое ядро с определенным шагом проходит по набору входных данных $(I)$ и вычисляет суммы произведений соответствующих значений ячеек и весов, формируя карту признаков $(I * K)$. Один сверточный слой может содержать несколько ядер и соответственно несколько карт признаков.

\begin{figure}[H]
\centering
\begin{tikzpicture}

	\matrix (mtr) [matrix of nodes,row sep=-\pgflinewidth, nodes={draw}]
	{
		0 & 1 & 1 & |[fill=red!30]| 1 & |[fill=red!30]| 0 & |[fill=red!30]| 0 & 0\\
		0 & 0 & 1 & |[fill=red!30]| 1 & |[fill=red!30]| 1 & |[fill=red!30]| 0 & 0\\
		0 & 0 & 0 & |[fill=red!30]| 1 & |[fill=red!30]| 1 & |[fill=red!30]| 1 & 0\\
		0 & 0 & 0 & 1 & 1 & 0 & 0\\
		0 & 0 & 1 & 1 & 0 & 0 & 0\\
		0 & 1 & 1 & 0 & 0 & 0 & 0\\
		1 & 1 & 0 & 0 & 0 & 0 & 0\\
	};

	\draw[very thick, red] (mtr-1-4.north west) rectangle (mtr-3-6.south east);

	\node [below= of mtr-5-4.south] (lm) {$\bf I$};

	\node[right = 0.2em of mtr] (str) {$*$};

	\matrix (K) [right=0.2em of str,matrix of nodes,row sep=-\pgflinewidth, nodes={draw, fill=blue!30}]
	{
		1 & 0 & 1 \\
		0 & 1 & 0 \\
		1 & 0 & 1 \\
	};
	\node [below = of K-3-2.south] (lk) {$\bf K$};

	\node [right = 0.2em of K] (eq) {$=$};

	\matrix (ret) [right=0.2em of eq,matrix of nodes,row sep=-\pgflinewidth, nodes={draw}]
	{
		1 & 4 & 3 & |[fill=green!30]| 4 & 1\\
		1 & 2 & 4 & 3 & 3\\
		1 & 2 & 3 & 4 & 1\\
		1 & 3 & 3 & 1 & 1\\
		3 & 3 & 1 & 1 & 0\\
	};
	\node [below = of ret-4-3.south] (lim) {${\bf I} * {\bf K}$};

	\draw[very thick, green] (ret-1-4.north west) rectangle (ret-1-4.south east);

	\draw[densely dotted, blue, thick] (mtr-1-4.north west) -- (K-1-1.north west);
	\draw[densely dotted, blue, thick] (mtr-3-4.south west) -- (K-3-1.south west);
	\draw[densely dotted, blue, thick] (mtr-1-6.north east) -- (K-1-3.north east);
	\draw[densely dotted, blue, thick] (mtr-3-6.south east) -- (K-3-3.south east);

	\draw[densely dotted, green, thick] (ret-1-4.north west) -- (K-1-1.north west);
	\draw[densely dotted, green, thick] (ret-1-4.south west) -- (K-3-1.south west);
	\draw[densely dotted, green, thick] (ret-1-4.north east) -- (K-1-3.north east);
	\draw[densely dotted, green, thick] (ret-1-4.south east) -- (K-3-3.south east);

	\matrix (K) [right=0.2em of str,matrix of nodes,row sep=-\pgflinewidth, nodes={draw, fill=blue!10}]
	{
		1 & 0 & 1 \\
		0 & 1 & 0 \\
		1 & 0 & 1 \\
	};

	\draw[very thick, blue] (K-1-1.north west) rectangle (K-3-3.south east);

\end{tikzpicture}
\caption{Операция свертки} \label{convolution}
\end{figure}

Так как признаки уже обнаружены, для упрощения дальнейших вычислений можно снизить детализацию входных данных. Это обеспечивает субдискретизирующий (пулинговый) слой, уменьшая размерность входных карт признаков: из нескольких соседних нейронов берется максимальное или среднее значение, тем самым формируя нейрон карты признаков меньшей размерности. Это позволяет снизить количество параметров, используемых в дальнейших вычислениях сети. 

\begin{figure}[H]
    \centering
    \begin{tikzpicture}
        \draw[step=1cm,very thin] (-2,-2) grid (2,2);
        \fill[blue, opacity=0.5] (0,2) rectangle (-2,0);
        \draw[thick,->] (3,0) -- (4,0) node[anchor=north west] {};
        \draw[step=1cm,very thin] (5,-1) grid (7,1);
        \fill[blue, opacity=0.5] (5,0) rectangle (6,1);
    \end{tikzpicture}
    \caption{Субдискретизация} \label{pooling}
\end{figure}

Сверточная нейронная сеть может иметь несколько пар чередующихся сверточных и субдискретизирующих слоев. 
Таким образом, на примере изображений, на начальных слоях сеть находит такие простейшие признаки как границы и углы. Затем, по мере углубления в сеть, определяются всё более сложные конструкции: от простейших фигур до целых классов, независимо от их местоположения на изображении. Завершается сеть стандартными полносвязными слоями которые сопоставляют полученные карты признаков какому-либо классу.  

\subsection{VGG}
Архитектура VGG была предложена в 2014 году\cite{simonyan2014convolutional}. Главной особенностью сети стало использование подряд стоящих сверточных слоев с фильтрами размерности 3x3 вместо применяемых ранее сверточных слоев с фильтрами большого размера 5x5, 7x7, 11x11. Это позволило снизить количество параметров сети с сохранением эффективности.

В таблице \ref{vgg} указаны различные конфигурации VGG, наиболее известные из них VGG-16 (D) и VGG-19 (E), названные по количеству слоев, содержащих веса. Maxpool - субдискретизирующий слой с функцией максимума размера 2x2. FC - полносвязный слой. Во всех скрытых слоях используется активационная функция ReLU. 

\begin{table}[H]
    \centering
    \caption{Конфигурации VGG} \label{vgg}
    \begin{tabular}{|c|c|c|c|c|c|} \hline
    A & A-LRN & B & C & D & E \\ \hline
    \multicolumn{6}{|c|}{Вход ($224 \times 224$ RGB)} \\ \hline
    conv3-64 & conv3-64 & conv3-64 & conv3-64 & conv3-64 & conv3-64 \\ 
     & \textbf{LRN} & \textbf{conv3-64} & conv3-64 & conv3-64 & conv3-64\\ \hline
    \multicolumn{6}{|c|}{Maxpool} \\ \hline
    conv3-128 & conv3-128 & conv3-128 & conv3-128 & conv3-128 & conv3-128 \\ 
     & & \textbf{conv3-128} & conv3-128 & conv3-128 & conv3-128 \\ \hline
    \multicolumn{6}{|c|}{Maxpool} \\ \hline
    conv3-256 & conv3-256 & conv3-256 & conv3-256 & conv3-256 & conv3-256 \\ 
    conv3-256 & conv3-256 & conv3-256 & conv3-256 & conv3-256 & conv3-256 \\ 
    & & & \textbf{conv1-256} & \textbf{conv3-256} & conv3-256 \\ 
    & & & & & \textbf{conv3-256} \\ \hline
    \multicolumn{6}{|c|}{Maxpool} \\ \hline
    conv3-512 & conv3-512 & conv3-512 & conv3-512 & conv3-512 & conv3-512 \\ 
    conv3-512 & conv3-512 & conv3-512 & conv3-512 & conv3-512 & conv3-512 \\ 
    & & & \textbf{conv1-512} & \textbf{conv3-512} & conv3-512 \\ 
    & & & & & \textbf{conv3-512} \\ \hline
    \multicolumn{6}{|c|}{Maxpool} \\ \hline
    conv3-512 & conv3-512 & conv3-512 & conv3-512 & conv3-512 & conv3-512 \\ 
    conv3-512 & conv3-512 & conv3-512 & conv3-512 & conv3-512 & conv3-512 \\ 
    & & & \textbf{conv1-512} & \textbf{conv3-512} & conv3-512 \\ 
    & & & & & \textbf{conv3-512} \\ \hline
    \multicolumn{6}{|c|}{Maxpool} \\ \hline
    \multicolumn{6}{|c|}{FC-4096} \\ \hline
    \multicolumn{6}{|c|}{FC-4096} \\ \hline
    \multicolumn{6}{|c|}{FC-1000} \\ \hline
    \multicolumn{6}{|c|}{Softmax} \\ \hline
    \end{tabular}
\end{table}

\subsection{Inception}
Данная модель\cite{1512.00567}, разработанная компанией Google, в 2014 году заняла 1 место в ежегодном конкурсе по классификации изображений - ILSVRC. Ключевым нововведением данной сети стало использование в качестве слоев вложенных модулей, которые представляют из себя набор фильтров разных размерностей, с последующим объединением их результатов. 

\begin{figure}[H]
    \centering
    \begin{tikzpicture}
        [align=center]
        \tikzstyle{rectangle_style}=[rectangle, draw, minimum height = 15mm, minimum width = 10mm]
        \tikzstyle{arrow} = [thick,->,>=stealth]
        \node (in) at (0,0) [rectangle_style] {Предыдущий слой};
        \node (cv1_1) at (-1.5,-2.5) [rectangle_style] {Conv 1x1};
        \node (cv3_1) at (-1.5,-5) [rectangle_style] {Conv 3x3};
        \node (cv1_2) at (2.0, -2.5) [rectangle_style] {Conv 1x1};
        \node (cv5_1) at (2.0,-5) [rectangle_style] {Conv 5x5};
        \node (mp_1)  at (5,-2.5) [rectangle_style] {Maxpool\\3x3};
        \node (cv1_3) at (5,-5) [rectangle_style] {Conv 1x1};
        \node (cv1_0) at (-5.5, -3.75) [rectangle_style] {Conv 1x1};
        \node (out) at (0,-7.5) [rectangle_style] {Объединение фильтров};

        \draw [arrow] (in) -- (cv1_1);
        \draw [arrow] (in) -- (cv1_2);
        \draw [arrow] (in) -- (cv1_0);
        \draw [arrow] (in) -- (mp_1);
        \draw [arrow] (cv1_1) -- (cv3_1);
        \draw [arrow] (cv1_2) -- (cv5_1);
        \draw [arrow] (mp_1) -- (cv1_3);
        \draw [arrow] (cv1_3) -- (out);
        \draw [arrow] (cv5_1) -- (out);
        \draw [arrow] (cv3_1) -- (out);
        \draw [arrow] (cv1_0) -- (out);

    \end{tikzpicture}
    \caption{Inception модуль} \label{inception-module}
\end{figure}

Также, в Inception полностью отказались от использования полносвязных слоев, вместо них применяется глобальный средний пулинг, который преобразует каждую карту признаков к одному числу, формируя вектор усредненных значений. Такое нововведение позволило значительно уменьшить количество параметров и как следствие вычислительную сложность сети. В последствии были разработаны улучшенные версии Inception, в которых заменили слой 5x5 двумя последовательными слоями 3x3, а также все слои c фильтрами размера NxN заменили на стек фильтров 1xN и Nx1, что также позволило снизить количество параметров.

\subsection{ResNet}
ResNet\cite{ResNet}, также известная как остаточная нейронная сеть, выйграла ILSVRC в 2015 году. Её особенностью было наличие пропускающих соединений, которые передают информацию без изменений на более глубокие участки сети, эта информация суммируется с вычисленным на пропущенных слоях значением и передается дальше. Блок изображенный на рис. \ref{res-net} демонстрирует составной элемент такой сети.

\begin{figure}[H]
\centering
    \begin{tikzpicture}
        \tikzset{
            connector/.style={
            -latex,
            font=\scriptsize
            },
            rectangle connector/.style={
                connector,
                to path={(\tikztostart) -- ++(#1,0pt) \tikztonodes |- (\tikztotarget) },
                pos=0.5
            },
            rectangle connector/.default=-2cm,
            straight connector/.style={
                connector,
                to path=--(\tikztotarget) \tikztonodes
            }
        }
        [align=center]
        \tikzstyle{rectangle_style}=[rectangle, draw, minimum height = 15mm, minimum width = 10mm]
        \tikzstyle{arrow} = [thick,->,>=stealth]
        \node (in) at (0,.3) [] {};
        \node (midin) at (0,-0.6) [] {};
        \node (cv3_1) at (0,-2) [rectangle_style] {Conv 3x3};
        \node (relu2) at (-0.8,-3.5) [] {$ReLU$};
        \node (cv3_2) at (0,-5) [rectangle_style] {Conv 3x3};
        \node (plus) at (0,-7) [circle, draw] {+};
        \node (out) at (0,-8.5) [] {};
        \node (x) at (0,.5) [] {$x$};
        \node (x2) at (-2.5,-4) [] {$x$};
        \node (f_x) at (2,-4) [] {$F(x)$};
        \node (H_x) at (1.5,-7) [] {$F(x)+x$};
        \node (relu2) at (- 0.8,-8) [] {$ReLU$};

        \draw [arrow, rectangle connector] (midin) to  node[] {} (plus);
        \draw [arrow] (in) -- (cv3_1);
        \draw [arrow] (cv3_1) -- (cv3_2);
        \draw [arrow] (cv3_2) -- (plus);
        \draw [arrow] (plus) -- (out);

    \end{tikzpicture}
\caption{Блок остаточной сети} \label{res-net}
\end{figure}

\subsection{DenseNet}
DenseNet\cite{DenseNet} - плотная сверточная сеть, похожая на ResNet, но с той разницей, что все блоки сети соединены прямыми связями между собой, таким образом каждый блок получает информацию от всех предыдущих. 

\newcommand{\conv}[1]{$\left[\begin{array}{ll} \text{1}\times \text{1} \text{ conv}\\ \text{3}\times \text{3} \text{ conv} \end{array}\right] \times \text{#1}$}
\newcommand{\cross}[1]{#1 $\times$ #1}
\begin{table}[H]
    \centering
    \caption{Архитектуры DenseNet}
    \label{densenets}
    \resizebox{0.845\textwidth}{!}{%
    \begin{tabular}{|c|c|c|l|l|l|}
        \hline
        Layers                                                                          & Output Size     & DenseNet-121                                                                                                               & \multicolumn{1}{c|}{DenseNet-169}                                                                     & \multicolumn{1}{c|}{DenseNet-201}                                                                     & \multicolumn{1}{c|}{DenseNet-264}                                                                       \\ \hline
        Convolution                                                                     & \cross{112} & \multicolumn{4}{c|}{\cross{7} conv, stride 2}                                                                                                                                                                                                                                                                                                                                                                                                    \\ \hline
        Pooling                                                                         & \cross{56}   & \multicolumn{4}{c|}{\cross{3} max pool, stride 2}                                                                                                                                                                                                                                                                                                                                                                                                \\ \hline
        \begin{tabular}[c]{@{}c@{}}Dense Block\\ (1)\end{tabular}                       & \cross{56}   & \multicolumn{1}{l|}{\conv{6}}  & \conv{6}  & \conv{6}  & \conv{6} \\ \hline
        \multirow{2}{*}{\begin{tabular}[c]{@{}c@{}}Transition Layer\\ (1)\end{tabular}} & \cross{56}  & \multicolumn{4}{c|}{\cross{1} conv}                                                                                                                                                                                                                                                                                                                                                                                                              \\ \cline{2-6}
                                                                                        & \cross{28}   & \multicolumn{4}{c|}{\cross{2} average pool, stride 2}                                                                                                                                                                                                                                                                                                                                                                                             \\ \hline
        \begin{tabular}[c]{@{}c@{}}Dense Block\\ (2)\end{tabular}                       & \cross{28}   & \multicolumn{1}{l|}{\conv{12}} & \conv{12}& \conv{12} & \conv{12} \\ \hline
        \multirow{2}{*}{\begin{tabular}[c]{@{}c@{}}Transition Layer\\ (2)\end{tabular}} & \cross{28}  & \multicolumn{4}{c|}{\cross{1} conv}                                                                                                                                                                                                                                                                                                                                                                                                              \\ \cline{2-6}
                                                                                        & \cross{14}   & \multicolumn{4}{c|}{\cross{2} average pool, stride 2}                                                                                                                                                                                                                                                                                                                                                                                             \\ \hline
        \begin{tabular}[c]{@{}c@{}}Dense Block\\ (3)\end{tabular}                       & \cross{14}   & \multicolumn{1}{l|}{\conv{24}} & \conv{32} & \conv{48} & \conv{64} \\ \hline
        \multirow{2}{*}{\begin{tabular}[c]{@{}c@{}}Transition Layer\\ (3)\end{tabular}} & \cross{14}   & \multicolumn{4}{c|}{\cross{1} conv}                                                                                                                                                                                                                                                                                                                                                                                                              \\ \cline{2-6}
                                                                                        & \cross{7}     & \multicolumn{4}{c|}{\cross{2} average pool, stride 2}                                                                                                                                                                                                                                                                                                                                                                                             \\ \hline
        \begin{tabular}[c]{@{}c@{}}Dense Block\\ (4)\end{tabular}                       & \cross{7}   & \multicolumn{1}{l|}{\conv{16}} & \conv{32} & \conv{32} & \conv{48}  \\ \hline
        \multirow{2}{*}{\begin{tabular}[c]{@{}c@{}}Classification\\ Layer\end{tabular}} & \cross{1} & \multicolumn{4}{c|}{\cross{7} global average pool}                                                                                                                                                                                                                                                                                                                                                                                            \\ \cline{2-6}
                                                                                        &                 & \multicolumn{4}{c|}{1000D fully-connected, softmax}                                                                                                                                                                                                                                                                                                                                                                                               \\ \hline
    \end{tabular}
    }
\end{table}



% \subsection{Xception}

\subsection{Тестирование}
Все указанные модели тестировались на наборах данных ImageNet, который включает в себя 1000 классов, свыше 1.3 млн. тренировочных и 50 тыс. проверочных изображений. Так как сети выполняют задачу классификации и используют на конце Softmax-слой, результатом сети является вектор $(x_1,x_2,...x_n)$, где n - количество классов, а $x_i$ - вероятность отношения входного изображения к i-му классу. Toчность сетей измерялась в двух вариантах Top-1 и Top-5. Top-1 означает, что наибольшее $x_i$ соответствует правильному классу, а Top-5 - что правильный класс принадлежит пяти самых высоким значениям в выходном векторе сети. В таблице \ref{conv-test} указаны результаты проведенных проверок по метрике accuracy - отношение доли правильных ответов к общему их количеству. 

\begin{table}[H]
    \centering
    \caption{Результаты тестирования моделей} \label{conv-test}
    \begin{tabular}{|c|c|c|c|}
      \hline    
      Сеть         & Кол-во параметров   & Top-1     & Top-5     \\
      \hline
      VGG-16       & 138 357 544         & 71.3\%	& 90.1\%    \\
      \hline
      VGG-19       & 143 667 240         & 71.3\%	& 90.0\%    \\
      \hline
      Inception V3 & 23 851 784          & 77.9\%	& 93.7\%    \\
      \hline
      ResNet-50 V2 & 25 613 800          & 76.0\%	& 93.0\%    \\
      \hline
      DenseNet-121 & 8 062 504           & 75.0\%	& 92.3\%    \\
      \hline
      DenseNet-169 & 14 307 880          & 76.2\%	& 93.2\%    \\
      \hline
      DenseNet-201 & 20 242 984          & 77.3\%	& 93.6\%    \\
      \hline
    \end{tabular}
  \end{table}

Как можно видеть на таблице: сети VGG имеют самое большое количество параметров, значительно превосходя другие модели, при этом демонстрируют самую низкую точность. Сети DenseNet показывают лучшее соотношение точности и количества параметров. Однако третья версия Inception имеет относительно немного больше параметров и точность. Лучший результат показала ResNet-152 второй версии, но она обладает значительно большим количеством параметров, что сказывается на времени обучения сети.

\clearpage