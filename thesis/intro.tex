\anonsection{Введение}
В настоящее время, в связи со стремительным развитием цифровых технологий, использование автоматизированных и роботизированных систем распространилось на множество областей как в промышленности, науке, так и в повседневной жизни. В следствие этого, возрастает необходимость в эффективной обработке информации, представленной, в частности, в формате видео и изображений. 

На текущий момент изображения тесно влились в жизнь человека. Поэтому многие автоматизированные системы используют их в качестве основного источника информации. Нахождение, локализация, классификация и анализ объектов на изображении компьютером – сложная задача компьютерного зрения. 

Компьютерное (машинное) зрение (Computer Vision) – это совокупность программно-технических решений в сфере искусственного интеллекта (ИИ), нацеленных на считывание и получение информации из изображений, в реальном времени и без участия человека. 

В процессе обработки информации, получаемой из глаз, человеческий мозг проделывает колоссальный объем работы. Человек без труда сможет описать что находится и что происходит на случайно взятой фотографии. Изображения могут нести в себе колоссальное количество деталей и отличаться множеством параметров, таких как: разрешение, цветность, качество, яркость, наличие шума и т.д. Объекты на изображениях также могут обладать множеством особенностей: масштаб, положение, цвет, поворот, наклон и т.д. Однако, в цифровом формате, каждое изображение представляет собой лишь массив числовых данных.  Научить компьютер находить и классифицировать образы на изображении с учетом всех факторов – очень сложная алгоритмическая задача. Для её решения активно применяют технологии машинного обучения.

Большое количество информации человек получает при помощи зрения. 
Изображения способны хранить огробное количество информации. Как следствие их использование в компьютерных системах способствует увеличению их производительности. Однако такие технологии требуют сложных вычислений. Задачей компьютерного зрения является разработка эффективных алгоритмов, выполняющих извлечение и анализ данных из изображений или видео. 
 
В настоящий момент, подобные технологии применяются для решения таких сложных задач как:
\begin{itemize}
    \item OCR – Optical character recognition (Оптическое распознавания символов): преобразование текста на изображении в редактируемый.
    \item Фотограмметрия – технология создания трехмерной модели объекта на основе фотографий, сделанных с различных ракурсов.
    \item Motion capture – технология, широко применяемая в киноиндустрии, позволяющая преобразовывать движения реальных людей в компьютерную анимацию.
    \item Дополненная реальность (AR) – технология, позволяющая в реальном времени проецировать виртуальные объекты на изображение реального окружения. 
    \item Медицинская диагностика – обнаружение раковых клеток на ранней стадии, увеличение качества МРТ изображений, их анализ и т.д.
\end{itemize}

В данной работе был проведен анализ алгоритмов глубокого машинного обучения для решения задач распознавания изображений. ???

Первая глава содержит основные сведения об машинном обучении.

Во второй главе проведено проектирование и реализация алгоритмов для распознавания изображений.

В третьей главе выполнен сравнительный анализ работы разработанных алгоритмов.

\clearpage