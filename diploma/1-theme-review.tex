\section{Обзор предметной области}

\subsection{Задачи компьютерного зрения}

\subsection{Искусственные нейронные сети}

\subsubsection{Понятие искусственной нейронной сети}

Машинное обучение – раздел исследований в сфере ИИ, в основе которых лежат методы разработки систем способных к обучению.

Искусственная нейронная сеть (ИНС) – компьютерная модель, в основе которой лежит упрощенное представление человеческого мозга. Структурной единицей ИНС является нейрон. 

\addimghere{simple-network}{0.5}{Схема простой нейронной сети}{simple-network}


\subsubsection{Архитектура нейронных сетей}

\subsubsection{Глубокие нейронные сети}

\subsubsection{Проблемы обучения нейронных сетей}

\subsection{Применение глубоких нейронных сетей в задачах распознавания изображений}

\clearpage