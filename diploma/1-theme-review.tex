\section{Обзор предметной области}
\subsection{Задача компьютерного зрения}
Компьютерное (машинное) зрение – это совокупность программно-технических решений в сфере искусственного интеллекта (ИИ), нацеленных на считывание и получение информации из изображений, в реальном времени и без участия человека. 

Большое количество информации человек получает при помощи зрения. 

В основе компьютерного зрения лежит 
В настоящий момент, такие технологии применяются для решения таких сложных задач как:
\begin{itemize}
    \item OCR – Optical character recognition (Оптическое распознавания символов): преобразование текста на изображении в редактируемый.
    \item Фотограмметрия – технология создания трехмерной модели объекта на основе фотографий, сделанных с различных ракурсов.
    \item Motion capture – технология, широко применяемая в киноиндустрии, позволяющая преобразовывать движения реальных людей в компьютерную анимацию.
    \item Дополненная реальность (AR) – технология, позволяющая в реальном времени проецировать виртуальные объекты на изображение реального окружения. 
    \item Медицинская диагностика – обнаружение раковых клеток на ранней стадии, увеличение качества МРТ изображений, их анализ и т.д.
\end{itemize}
\subsection{Искусственные нейронные сети}

\subsubsection{Понятие искусственной нейронной сети}

Машинное обучение – раздел исследований в сфере ИИ, в основе которых лежат методы разработки систем способных к обучению.

Искусственная нейронная сеть (ИНС) – компьютерная модель, в основе которой лежат принципы работы биологической нейронной сети - совокупность связанных между собой нервных клеток - нейронов. Каждый нейрон имеет набор входных связей - синапсов, по которым он получает информацию, представленную в виде импульсов, от других нейронов. По полученным данным нейрон формирует своё состояние и с помощью аксона сообщает его другим нейронам, обеспечивая функционирование системы. 
\addimghere{biological-neuron}{0.5}{Типичная структура нейрона}{biological-neuron}
Искусственный нейрон представляет собой упрощенную модель биологического нейрона. На входе нейрон получает n-мерный вектор значений \(X=(x_{1},...,x_{n})\) и вектор весов \(W=(w_{1},...,w_{n})\). Значение выхода нейрона вычисляется по формуле: 
\[
  out(x)=\sigma(\sum_{\mathclap{1\le i\le n}}^{n} x_{i}w_{i})
\]
Где \(\sigma\) - функция активации.

\subsubsection{активационная функция}
Активационная функция нейрона обеспечивает нормализацию посчитанной суммы. 
\subsubsection{Глубокие нейронные сети}

\subsubsection{Сверточные нейронные сети}

\subsubsection{Проблемы обучения нейронных сетей}

\subsection{Применение нейронных сетей в задачах распознавания изображений}

\addimghere{simple-network}{0.5}{Схема простой нейронной сети}{simple-network}
\clearpage